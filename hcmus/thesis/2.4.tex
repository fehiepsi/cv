\begin{center}
\textbf{2.4. BIỂU DIỄN GREEN}
\end{center}

\noindent\textbf{\underline{Một số công thức:}}
\\
\textit{Cho miền mở,liên thông,bị chặn $\Omega_{0}$ sao cho $\partial{\Omega _{0}} \in C^{1}$.Ký hiệu
$\nu$ là véc tơ pháp tuyến đơn vị của $\partial{\Omega_{0}}$ hướng ra ngoài.Khi này với mọi trường véc tơ $w$ trong $ C^1(\overline {\Omega _0 })$
thì ta có được điều sau:}
\[
\int\limits_{\Omega _0 } {div(w)dx}  = \int\limits_{\partial \Omega _0 } {w.} \nu ds\]
\textit{trong đó $ds$ biểu thị vùng diện tích $(n-1)$ chiều của $\partial{\Omega_{0}}$}
\\
\\
\textbf{I) \underline{Bây giờ chúng ta sẽ đi vào nghiên cứu những công thức của Green:}}

Ta quy định miền $\Omega$ như miền $\Omega_{0}$ trong định lý phía trên,khi này ta xét $u,v$ là hai hàm $C^2 (\overline \Omega  )$. Chúng ta thế $w=vDu$ vào định lý trên chúng ta sẽ thu được điều sau:
\begin{center}
\textbf{$\int\limits_\Omega  {v\Delta udx + \int\limits_\Omega  {Du.Dvdx = \int\limits_{\partial \Omega } {v.\frac{{\partial u}}
{{\partial \nu }}} } } ds$       (2.4.1)}
\end{center}

Biểu thức (2.4.1) được gọi là \textbf{công thức Green thứ nhất}.
\\
\textbf{Chứng minh công thức:}

Thật vậy:

Khi thế $w=vDu$ vào định lý trên thì chúng ta thu được:
\[
\int\limits_\Omega  {div(vDu)dx}  = \int\limits_{\partial \Omega } {(vDu).} \nu ds
\]

Ta có thể thấy ngay từ định nghĩa divergence rằng:
\[div(vDu) = Dv.Du + v.\Delta u\]

Nên: 
\[
\int\limits_\Omega  {div(vDu)dx}  = \int\limits_\Omega  {v\Delta udx + \int\limits_\Omega  {Du.Dvdx} } 
\]

Đồng thời, do $
Du.\nu  = \frac{{\partial u}}
{{\partial \nu }}
$ nên ta có $
\int\limits_{\partial \Omega } {(vDu).\nu ds}  = \int\limits_{\partial \Omega } {v.\frac{{\partial u}}
{{\partial \nu }}} ds.
$

Từ đây ta thu được công thức trên.

Bây giờ,ta cũng xét cùng những điều kiện như công thức Green thứ nhất.Khi này ta có công thức sau:
\begin{center}
\textbf{$
\int\limits_\Omega  {(v\Delta u - u\Delta v)dx = \int\limits_{\partial \Omega } {(v} } \frac{{\partial u}}
{{\partial \nu }} - u\frac{{\partial v}}
{{\partial \nu }})ds
$ (2.4.2)}
\end{center}

Công thức (2.4.2) được gọi là \textbf{công thức Green thứ hai}.

Lời giải cho công thức trên khá đơn giản vì ta chỉ cần hoán đổi vai trò của $u,v$ trong công thức Green thứ nhất(bởi vì vai trò của $u,v$ ở đây là như nhau).Rõ ràng lúc này toán hạng $
{\int\limits_\Omega  {Du.Dvdx} }$ không thay đổi nên khi ta đem trừ hai đẳng thức sau khi đã hoán đổi vai trò $u,v$ thì ta thu được ngay công thức Green thứ hai.
\\
\\
\textbf{II) \underline{Một số tính chất của nghiệm cơ bản phương trình Laplace:}}\\
Đầu tiên ta nhắc lại định nghĩa của nghiệm cơ bản phương trình Laplace:
\[
\Gamma (x - y) = \Gamma (|x - y|) = \left\{ {\begin{array}{*{20}c}
   {\frac{1}
{{n(2 - n)\omega _n }}|x - y|^{(2 - n)} \quad n > 2}  \\
   {\frac{1}
{{2\pi }}\ln |x - y|\quad \quad \quad \quad \quad n = 2}  \\

 \end{array} } \right.\quad 
\]  
\textbf{Ký hiệu đẳng thức trên là (2.4.3)} và ở đây ta cố định $y \in \Omega$, $\omega_{n}$ là thể tích quả cầu đơn vị $n$ chiều.
\\
\textbf{\underline{Tính chất 1:}}
\\
Nghiệm cơ bản của phương trình Laplace là hàm điều hòa.\\
\textbf{Chứng minh:}
\\
\textbf{\underline{Xét đạo hàm bậc nhất:}}

	Xét 2 trường hợp:\\
		+Khi $n=2$:
\[
D_i \Gamma (x - y) = \frac{\partial }
{{\partial x_i }}(\frac{1}
{{2\pi }}\ln |x - y|) = \frac{1}
{{2\pi }}.\frac{{\frac{\partial }
{{\partial x_i }}(|x - y|)}}
{{|x - y|}}\quad 
\]

Rõ ràng: $
\frac{\partial }
{{\partial x_i }}(|x - y|) = (x_i  - y_i ).|x - y|^{ - 1} 
$ nên:
\[
D_i \Gamma (x - y) = \frac{1}
{{2\pi }}.(x_i  - y_i )|x - y|^{ - 2} 
\]

Khi $n=2$ thì thể tích của quả cầu đơn vị khi này là $\pi$ nên từ đây:
\[
D_i \Gamma (x - y) = \frac{1}
{{2\omega _2 }}.(x_i  - y_i )|x - y|^{ - 2} 
\]
+Khi $n>2$. Lúc này:
\[
D_i \Gamma (x - y) = \frac{1}
{{n(2 - n)\omega _n }}\frac{\partial }
{{\partial x_i }}(|x - y|^{(2 - n)} )
\]

Bằng lập luận tương tự như trường hợp $n=2$ chúng ta thu được:
\[
\frac{\partial }
{{\partial x_i }}(|x - y|^{(2 - n)} ) = (2 - n)(x_i  - y_i )|x - y|^{ - n} 
\]

Do đó:
\[
D_i \Gamma (x - y) = \frac{1}
{{n\omega _n }}(x_i  - y_i )|x - y|^{ - n} 
\]

Chung lại,sau 2 trường hợp ta thu được:
\[D_i \Gamma (x - y) = \frac{1}
{{n\omega _n }}(x_i  - y_i )|x - y|^{ - n} \quad (\forall n  \geqslant 2
)\]
\textbf{\underline{Xét đạo hàm cấp 2:}}

 Làm tương tự như trường hợp đạo hàn cấp 1,chúng ta thu được công thức cho đạo hàm cấp 2 như sau:
\[D_{ij} \Gamma (x - y) = \frac{1}
{{n\omega _n }}(|x - y|^2 \delta _{ij}  - n(x_i  - y_i )(x_j  - y_j ))|x - y|^{ - n - 2} 
\]
trong đó $\delta _{ij}$ là ký hiệu Kronecker.\\
\textbf{\underline{Chứng minh $\Gamma$ là hàm điều hòa:}}

Bằng cách áp dụng công thức đạo hàm bậc 2,khi $x \ne y$ thì:
 \[
\begin{gathered}
  \Delta \Gamma (x - y) = \sum\limits_{i = 1}^n {D_{ii} } \Gamma (x - y) \hfill \\
  \quad \quad \quad \quad  = \sum\limits_{i = 1}^n {\frac{1}
{{n\omega _n }}[|x - y|^2  - n(x_i  - y_i )^2 ]|x - y|^{ - n - 2} }  \hfill \\
  \quad \quad \quad \quad  = \frac{1}
{{n\omega _n }}|x - y|^{ - n - 2} \sum\limits_{i = 1}^n {|x - y|^2  - n(x_i  - y_i )^2  = 0}  \hfill \\
\end{gathered} 
\]

Vậy $\Gamma$ là hàm điều hòa
\\
\\
\textbf{\underline{Tính chất 2:}}
\[
D^\beta  \Gamma (x - y) \le C|x - y|^{2 - n - |\beta |} 
\]
trong đó $C = C(n,|\beta |)$ và \textbf{ký hiệu bất đẳng thức này là (2.4.4)}\\
hay một cách tổng quát hơn
\[
|D^k u(x)| \leqslant \frac{{C_k }}
{{r^{n + k} }}\int_{B(x,r)} {|u|dy} 
\]
với $u$ là hàm điều hòa trong $ \overline { \Omega } $, với mọi quả cầu $B(x,r) \subset \overline { \Omega }$, $
C_0  = \frac{1}{{\omega _n }}
$,
$
C_k  = \frac{(2^{n + 1} nk)^k }
{\omega _n }$  $(k > 0)$. 

\textbf{Ký hiệu bất đẳng thức này là (2.4.5)}

Chúng ta thử giải bất đẳng thức (2.4.4) trong trường hợp đạo hàm bậc 1,2( tức là $\beta  = 1,2$ bằng cách bình thường để thấy ý nghĩa của nó.
\\
\textbf{Đối với đạo hàm bậc nhất}:

Áp dụng đẳng thức đạo hàm bậc nhất của $\Gamma$,ta có:
\[
\left| {D_i \Gamma (x - y)} \right| \le \frac{1}{{n\omega _n }}|x - y|^{(1 - n)} 
\]

Do công thức đạo hàm bậc nhất là $
D_i \Gamma (x - y) = \frac{1}
{{n\omega _n }}(x_i  - y_i )|x - y|^{ - n} \quad (\forall n  \geqslant 2
)
$ và $
(x_i  - y_i ) \le |x - y|
$
\\
\textbf{Đối với đạo hàm bậc hai}:

Áp dụng đẳng thức đạo hàm bậc hai của $\Gamma$,ta có:
\[
D_{ij} \Gamma (x - y) = \frac{1}{{n\omega _n }}[|x - y|^2 \delta _{ij}  - n(x_i  - y_i )(x_j  - y_j )]|x - y|^{ - n - 2} 
\]

Ta đi chứng minh rằng:
\[
\frac{1}{n}\left| {|x - y|^2 \delta _{ij}  - n(x_i  - y_i )(x_j  - y_j )} \right| \le |x - y|^2 \quad \quad \quad (*)
\]

Thật vậy (*) tương đương cần chứng minh:
\[
 - n|x - y|^2  \le n(x_i  - y_i )(x_j  - y_j ) - |x - y|^2 \delta _{ij}  \le n|x - y|^2 
\]
\[
\Leftrightarrow
 - (n + \delta _{ij} )|x - y|^2  \le n(x_i  - y_i )(x_j  - y_j ) \le (n + \delta _{ij} )|x - y|^2 
\]
\\
$\bullet$ Khi $i=j$ thì bất đẳng thức trên trở thành:
\[
 - (n + 1)|x - y|^2  \le n(x_i  - y_j )^2  \le (n + 1)|x - y|^2 
\]

Bất đẳng thức trên là đúng do $
(x_i  - y_i )^2  \le |x - y|^2 
$.\\
$\bullet$ Khi $i \neq j$. Lúc này bất đẳng thức trở thành:
\[
 - n|x - y|^2  \le n|(x_i  - y_i )(x_j  - y_j )| \le n|x - y|^2 
\]

Bất đẳng thức vế trái hiển nhiên đúng vì VP $\geq$ 0 còn VT $\leq$ 0.

Xét bất đẳng thức vế phải:
\[
n|(x_i  - y_i )(x_j  - y_j )| \le n|x - y|^2 
\;\Leftrightarrow
|(x_i  - y_i )(x_j  - y_j )| \le |x - y|^2 
\]

Điều trên là đúng vì:
\[
|x - y|^2  = \sum\limits_{i = 1}^n {(x_i }  - y_i )^2  \ge (x_i  - y_i )^2  + (x_j  - y_j )^2  \ge 2|(x_i  - y_i )(x_j  - y_j )| \ge |(x_i  - y_i )(x_j  - y_j )|
\]

Do đó bất đẳng thức (*) được chứng minh.

Áp dụng bất đẳng thức (*) vào đẳng thức đạo hàm cấp 2 của $\Gamma$ ta thu được:
\[|D_{ij} \Gamma (x - y)| \leqslant \frac{1}
{{\omega _n }}|x - y|^{ - n}\]
\textbf{Bây giờ thay vì chứng minh tính chất 2 thì ta chứng minh bất đẳng thức (2.4.5) vì bất đẳng thức (2.4.4) là trường hợp nhỏ của bất đẳng thức (2.4.5)}

Đầu tiên ta nhắc lại công thức trung bình đối với hàm điều hòa $u$ trong miền $\Omega$ như sau:
\[
u(x) = \frac{1}{{r^n \omega _n }}\int\limits_{B(x,r)} {udy} 
\]	
với mọi quả cầu $B(x,r) \subset \Omega $.

Bây giờ ta đi chứng minh bất đẳng thức (2.4.5) bằng phương pháp quy nạp theo $k$.\\
$\bullet$ Khi $k=0$ thì bất đẳng thức (2.4.5) trở thành:
\[
|u(x)| \le \frac{1}{{r^n \omega _n }}\int\limits_{B(x,r)} {|u|dy} 
\]

Thật vậy, do $u$ là hàm điều hòa nên:
\[
|u(x)| = |\frac{1}{{r^n \omega _n }}\int\limits_{B(x,r)} {udy} | \le \frac{1}{{r^n \omega _n }}\int\limits_{B(x,r)} {|u|dy} 
\]

$\bullet$ Khi $k=1$. Vì u là hàm điều hòa trong $\Omega$ nên các đạo hàm thành phần $D_i u$ cũng là các hàm điều hòa trong $\Omega$.

Khi này bằng cách áp dụng công thức giá trị trung bình cho hàm điều hòa, ta được:
\[
|D_i u| = |\frac{{2^n }}{{r^n \omega _n }}\int\limits_{B(x,\frac{r}{2})} {D_i udy} | = |\frac{{2^n }}{{r^n \omega _n }}\int\limits_{\partial B(x,\frac{r}{2})} {u\nu _i ds} |
\]

Nếu xét $x_0  \in \partial B(x,\frac{r}{2})$ thì $
B(x_0 ,\frac{r}{2}) \subset B(x,r) \subset U
$. Lúc này, bằng việc áp dụng mệnh đề quy nạp khi $k=0$:
\[
|u(x_0 )| \le \frac{{2^n }}{{r^n \omega _n }}\int\limits_{B(x_0 ,\frac{r}{2})} {|u|dy}  = \frac{{2^n }}{{r^n \omega _n }}||u||_{L^1 (B(x_0 ,\frac{r}{2}))}  \le \frac{{2^n }}{{r^n \omega _n }}||u||_{L^1 (B(x,r))} 
\]

Từ đây:
\[
|D_i u| = |\frac{{2^n }}{{r^n \omega _n }}\int\limits_{\partial B(x,\frac{r}{2})} {u\nu _i ds} | \le \frac{{2^n }}{{r^n \omega _n }}\frac{{2^n }}{{r^n \omega _n }}||u||_{L^1 (B(x,r))} \int\limits_{\partial B(x,\frac{r}{2})} ds \]
\[= \frac{{2^n }}{{r^n \omega _n }}\frac{{2^n }}{{r^n \omega _n }}||u||_{L^1 (B(x,r))} n\frac{{r^{n - 1} }}{{2^{n - 1} }} \omega _n  = \frac{{2^{n + 1} n}}{{r^{n + 1} \omega _n }}||u||_{L^1 (B(x,r))}  = \frac{{C_1 }}{{r^{n + 1} }}||u||_{L^1 (B(x,r))} .
\]

Do đó mệnh đề quy nạp được chứng minh cho trường hợp $k=1$.\\
$\bullet$ Khi $k \geq 2$.Giả sử khi này bất đẳng thức chúng ta chứng minh đã đúng cho $(k-1)$.

Xét $\alpha$ là đa chỉ số có bậc là $k$. Khi này $
D_\alpha  u = (D_\beta  u)_{x_i } $ với $i=(1,2,3,...,n)$,$|\beta|=k-1$. Đồng thời, các $D_\alpha u$ cũng là các hàm điều hòa trong $\Omega$ nên ta có thể áp dụng công thức giá trị trung bình:
\[
|D_\alpha  u(x)| = |\frac{{k^n }}{{r^n \omega _n }}\int\limits_{B(x,\frac{r}{k})} {D_\alpha  udy} | = |\frac{{k^n }}{{r^n \omega _n }}\int\limits_{\partial B(x,\frac{r}{k})} {D_\beta  u\nu ds} |
\]

Nếu $x_0  \in \partial B(x,\frac{r}{k})$ thì $B(x_0 ,\frac{{r(k - 1)}}{k}) \subset B(x,r) \subset U$. Áp dụng giả thuyết quy nạp cho $(k-1)$, ta được:
\[
|D_\beta  u(x_0 )| \le \frac{{(2^{n + 1} n(k - 1))^{(k - 1)} }}{{\omega _n (\frac{{k - 1}}{k}r)^{n + k - 1} }}||u||_{L^1 (B(x_0 ,\frac{{k - 1}}{k}r))}  \le \frac{{(2^{n + 1} n(k - 1))^{(k - 1)} }}{{\omega _n (\frac{{k - 1}}{k}r)^{n + k - 1} }}||u||_{L^1 (B(x,r))} 
\]

Từ đây:
\[
|D_\alpha  u(x)| = |\frac{{k^n }}{{r^n \omega _n }}\int\limits_{\partial B(x,\frac{r}{k})} {D_\beta  u\nu ds} | \le \frac{{k^n }}{{r^n \omega _n }}\frac{{(2^{n + 1} n(k - 1))^{(k - 1)} }}{{\omega _n (\frac{{k - 1}}{k}r)^{n + k - 1} }}||u||_{L^1 (B(x ,r))} \int\limits_{\partial B(x,\frac{r}{k})} {ds}= 
\]
\[
 = \frac{{k^n }}{{r^n \omega _n }}\frac{{(2^{n + 1} n(k - 1))^{(k - 1)} }}{{\omega _n (\frac{{k - 1}}{k}r)^{n + k - 1} }}n\omega _n \left( {\frac{r}{k}} \right)^{n - 1} ||u||_{L^1 (B(x,r))}  = \frac{{(2^{n + 1} nk)^k }}{{\omega _n r^{n + k} }}||u||_{L^1 (B(x,r))}  = \frac{{C_k }}{{r^{n + k} }}||u||_{L^1 (B(x,r))} 
 \]

Nên mệnh đề quy nạp được chứng minh cho $k \geq 2$.

Vậy bất đẳng thức (2.4.5) được chứng minh hoàn toàn.
\\
\\
\textbf{III) \underline{Công thức biểu diễn Green và một số tính chất:}}

Ta thấy rằng $\Gamma$ kỳ dị tại $x=y$ nên ta sẽ tìm cách tránh điều trên bằng cách xét $\Omega \backslash B(y,r)$

Khi này bằng việc áp dụng công thức GREEN thứ nhất ta được:\\
$
\begin{gathered}
  \int\limits_{\Omega \backslash B(y,r)} {\Gamma \Delta udx = \int\limits_{\partial \Omega } {(\Gamma \frac{{\partial u}}
{{\partial \nu }}} }  - u\frac{{\partial \Gamma }}
{{\partial \nu }})ds + \int\limits_{\partial B(y,r)} {(\Gamma \frac{{\partial u}}
{{\partial \nu }}}  - u\frac{{\partial \Gamma }}
{{\partial \nu }})ds \hfill \\
  \quad \quad \quad \quad \;\quad  = \int\limits_{\partial \Omega } {(\Gamma \frac{{\partial u}}
{{\partial \nu }} - u\frac{{\partial \Gamma }}
{{\partial \nu }})ds + } I_1  - I_2 
 \hfill \\ 
\end{gathered} 
\quad \quad \quad  (2.4.6)$

\textbf{Xét $I_1$:}\\
\[
\int\limits_{\partial B(y,r)} {\Gamma \frac{{\partial u}}
{{\partial \nu }}ds = \Gamma (r)\int\limits_{\partial B(y,r)} {\frac{{\partial u}}
{{\partial \nu }}ds \leqslant n\omega _n r^{n - 1} \Gamma (r)\mathop {\sup }\limits_{B(y,r)} } } |Du|
\]

Thật vậy:
\[
|\int\limits_{\partial B(y,r)} {\frac{{\partial u}}{{\partial \nu }}} ds| = |\int\limits_{\partial B(y,r)} {Du.\nu } ds| \le \int\limits_{\partial B(y,r)} {|Du.\nu |ds \le \int\limits_{\partial B(y,r)} {|Du|.|\nu |ds} } 
\]

 Mà:
\[
\int\limits_{\partial B(y,r)} {|Du|.|\nu |ds}  = \int\limits_{\partial B(y,r)} {|Du|ds}  \le \mathop {\sup }\limits_{B(y,r)} |Du|\int\limits_{\partial B(y,r)} {ds}  = 
\]
 
\[
 = n\omega _n r^{n - 1} \mathop {\sup }\limits_{B^{'}(y,r)} |Du|
\]

Theo định nghĩa hàm $\Gamma$ có:
$
\Gamma (r) = \frac{{r^{2 - n} }}{{n(2 - n)\omega _n }}
$

Do đó:
$
n\omega _n r^{n - 1} \Gamma (r) = \frac{r}{{2 - n}}
$

Từ đẳng thức trên, ta có:
\[
{\mathop {\lim }\limits_{r \to 0} n\omega _n r^{n - 1} \Gamma (r)}{\mathop {\sup }\limits_{B^{'}(y,r)} }|Du| = 0
\]

Chung lại:
\[
n\omega _n r^{n - 1} \Gamma (r).\mathop {\sup }\limits_{B^{'}(y,r)} |Du| = \frac{r}{{2 - n}}\mathop {\sup }\limits_{B^{'}(y,r)} |Du| \to 0
\]

Hay:
\[
\mathop {\lim }\limits_{r \to 0} \int\limits_{\partial B(y,r)} {\Gamma \frac{{\partial u}}
{{\partial \nu }}} ds = 0
\]
\textbf{Xét $I_2$:}
\[
\int\limits_{\partial B(y,r)} {u(x)\frac{{\partial \Gamma }}
{{\partial \nu }}(x - y)ds} (x) = \int\limits_{\partial B(0,r)} {u(y + z)\frac{{\partial \Gamma }}
{{\partial \nu }}(z)ds} (z)
\]

Và ta chứng minh được rằng:
\[
\frac{{\partial \Gamma }}
{{\partial \nu }}(z) = D\Gamma (z)\nu  = \frac{{ - z}}
{{n\omega _n |z|^n }}.\frac{z}
{{|z|}} = \frac{{ - 1}}
{{n\omega _n |z|^{n - 1} }} = \frac{{ - 1}}
{{n\omega _n r^{n - 1} }}
\]

Nên:
\[
\int\limits_{\partial B(y,r)} {u(x)\frac{{\partial \Gamma }}
{{\partial \nu }}(x - y)ds} (x) = \frac{{ - 1}}
{{n\omega _n r^{n - 1} }}\int\limits_{\partial B(0,r)} {u(y + z)ds(z)} 
\]

Từ đẳng thức trên:
\[
\begin{gathered}
  \mathop {\lim }\limits_{r \to 0} \int\limits_{\partial B(y,r)} {u(x)\frac{{\partial \Gamma }}
{{\partial \nu }}(x - y)ds} (x) = \mathop {\lim }\limits_{r \to 0}  - \int\limits_{\partial B(0,r)}^ -  {u(y + z)ds(z)}  \hfill \\
  \quad \quad \quad \quad \quad  = \mathop {\lim }\limits_{r \to 0}  - \int\limits_{\partial B(y,r)}^ -  {u(x)ds(x)}  =  - u(y) \hfill \\ 
\end{gathered} 
\]

Từ đây nếu cho $r \to 0$ thì ở đẳng thức (2.4.6) ta thu được điều sau:
\[
\mathop {\lim }\limits_{r \to 0} \int\limits_{\Omega \backslash B(y,r)} {\Gamma \Delta udx = \int\limits_{\partial \Omega } {(\Gamma \frac{{\partial u}}{{\partial \nu }}} }  - u\frac{{\partial \Gamma }}{{\partial \nu }})ds - u(y)
\]

Ta đi chứng minh: 
\[
\mathop {\lim }\limits_{r \to 0} \int\limits_{\Omega \backslash B(y,r)} {\Gamma \Delta udx = \int\limits_\Omega  {\Gamma \Delta udx} } 
\]

Điều cần chứng minh phía trên tương đương với chứng minh:
\[
\mathop {\lim }\limits_{r \to 0} \int\limits_{B(y,r)} {\Gamma \Delta udx}  = 0
\]

Áp dụng biến đổi tọa độ cực:
\[
\int\limits_{B(y,r)} {\Gamma \Delta udx}  = \int\limits_0^r {(\int\limits_{\partial B(y,t)} {\Gamma \Delta ud} } s)dt
\]

Có:
\[
\int\limits_{\partial B(y,t)} {\Gamma \Delta ud} s = \frac{1}{{n(2 - n)\omega _n }}t^{2 - n} \int\limits_{\partial B(y,t)} {\Delta uds} 
\]

Nên:
\[
\int\limits_{B(y,r)} {\Gamma \Delta udx}  = \int\limits_0^r {(\int\limits_{\partial B(y,t)} {\Gamma \Delta ud} } s)dt = \frac{1}{{n(2 - n)\omega _n }}\int\limits_0^r {(t^{2 - n} \int\limits_{\partial B(y,t)} {\Delta uds} )dt} 
\quad \quad \quad  (2.4.7)
\]

Đồng thời,ta có:
\[
\int\limits_{\partial B(y,t)} {\Delta uds}  \le M\int\limits_{\partial B(y,t)} {ds}  = Mt^{n - 1} n\omega _n 
\quad \quad \quad  (2.4.8)
\]

Kết hợp (2.4.7) và (2.4.8):
\[
\int\limits_{B(y,r)} {\Gamma \Delta udx}  \le \frac{M}{{(2 - n)}}\int\limits_0^r {tdt = \frac{{Mr^2 }}{{2(2 - n)}} \to 0} 
\]

Do đó:
\[
\mathop {\lim }\limits_{r \to 0} \int\limits_{\Omega \backslash B(y,r)} {\Gamma \Delta udx = \int\limits_\Omega  {\Gamma \Delta udx} } 
\]

Từ:$\;\mathop {\lim }\limits_{r \to 0} \int\limits_{\Omega \backslash B(y,r)} {\Gamma \Delta udx = \int\limits_{\partial \Omega } {(\Gamma \frac{{\partial u}}{{\partial \nu }}} }  - u\frac{{\partial \Gamma }}{{\partial \nu }})ds - u(y)$ và $\;$ $\mathop {\lim }\limits_{r \to 0} \int\limits_{\Omega \backslash B(y,r)} {\Gamma \Delta udx = \int\limits_\Omega  {\Gamma \Delta udx} } $ nên khi cho $r \to 0$ ta thu được:
\[
u(y) = \int\limits_{\partial \Omega } {(u\frac{{\partial \Gamma }}
{{\partial \nu }}}  - \Gamma \frac{{\partial u}}
{{\partial \nu }})ds + \int\limits_\Omega  {\Gamma \Delta udx}
\]

Hay viết dưới dạng cụ thể hơn:
\[u(y) = \int\limits_{\partial \Omega } {(u\frac{{\partial \Gamma }}{{\partial \nu }}} (x - y) - \Gamma (x - y)\frac{{\partial u}}{{\partial \nu }})ds + \int\limits_\Omega  {\Gamma (x - y)\Delta udx} \quad \quad \quad \quad (2.4.9)\]
\textbf{Đẳng thức (2.4.9) được gọi là công thức biểu diễn Green}
\\
\textit{\underline{Một vài lưu ý liên quan đến công thức biểu diễn Green:}}
\\
$\quad \quad \quad$ $\bullet$ Nếu hàm $f$ khi này khả tích thì tích phân $\int\limits_\Omega  {\Gamma (x - y)\Delta udx} $ được gọi là "Newton potential" với mật độ là $\Delta u$.\\
$\bullet$ Nếu hàm $u$ có giá compact trên $R^{n}$ thì công thức (2.4.9) cho ta đẳng thức sau:
\[u(y) = \int\limits_{R^n } {\Gamma (x - y)\Delta u(x)dx} \]
$\bullet$ Nếu hàm $u$ là hàm điều hòa trong $\Omega$ thì khi này công thức (2.4.9) cho ta công thức sau:
\[
u(y) = \int\limits_{\partial \Omega } {(u\frac{{\partial \Gamma }}{{\partial \nu }}} (x - y) - \Gamma (x - y)\frac{{\partial u}}{{\partial \nu }})ds
\]
\\
\textbf{IV) \underline{Hàm Green:}}

Xét hàm  $h \in C^1 (\overline \Omega  ) \cap C^2 (\Omega )$ và $h$ là hàm điều hòa trong $\Omega$. Khi này bằng cách áp dụng công thức Green thứ hai(tức (2.4.2)) ta được:
\[
\int\limits_\Omega  {h\Delta u} dx = \int\limits_{\partial \Omega } {(h\frac{{\partial u}}
{{\partial \nu }}}  - u\frac{{\partial h}}
{{\partial \nu }})ds
\]

Đặt:$\;$ $G = \Gamma  + h$ thì khi này $G$ là hàm điều hòa trên $\Omega$ (vì hàm $\Gamma$ và $h$ điều hòa trên $\Omega$).Áp dụng công thức biểu diễn Green ở (2.4.9):\\
\[
u(y) = \int\limits_{\partial \Omega } {(u\frac{{\partial G}}{{\partial \nu }}}  - G\frac{{\partial u}}{{\partial \nu }})ds + \int\limits_\Omega  {G\Delta udx} 
\]

Thêm vào đó, nếu $G=0$ trên $\partial \Omega $ , nghĩa là khi này hàm $h$ có dạng như sau:
\[
\left\{ {\begin{array}{*{20}c}
   {\Delta h = 0\quad trong\,\Omega }  \\
   {h(x) =  - \Gamma (x - y)\quad tren \,\partial \Omega }  \\
\end{array}} \right.
\]
thì khi này công thức trên trở thành:
\[
u(y) = \int\limits_{\partial \Omega } {u\frac{{\partial G}}{{\partial \nu }}} ds + \int\limits_\Omega  {G\Delta udx} 
\]
	
Hàm $G=G(x,y)$ xác định như trên được gọi là hàm Green cho miền mở,liên thông,bị chặn $\Omega$. Hàm $G$ đôi khi còn được gọi là \textbf{Hàm Green loại 1 của miền $\Omega$}.

Nhắc lại định lý về tồn tại duy nhất của hàm điều hòa cho bài toán Dirichlet như sau:

\textit{Cho $u,v \in C^2 (\Omega ) \cap C^0 (\overline \Omega  )$ thỏa mãn $
\Delta u = \Delta v$ trong $\Omega$ và $u=v$ trên $\partial \Omega$. Lúc này $u=v$ trong $\Omega$.}

Bằng cách áp dụng định lý trên, do điều kiện $G=0$ trên $\partial \Omega $ nên hàm Green khi này là xác định duy nhất.
