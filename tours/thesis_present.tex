\documentclass{beamer}
\usepackage[utf8]{inputenc}
\usepackage[english]{babel}
\usepackage{amsmath, amssymb, amsthm, xcolor}
\usetheme{Madrid}
\usecolortheme{beaver}

%Định nghĩa màu cho chữ viết, các bạn không nên xóa bỏ phần này

\newcommand{\doo}[1]{\textcolor{red}{#1}}
\newcommand{\xanh}[1]{\textcolor{green}{#1}}
\newcommand{\duong}[1]{\textcolor{blue}{#1}}
\newcommand{\bich}[1]{\textcolor{cyan}{#1}}
\newcommand{\hong}[1]{\textcolor{magenta}{#1}}
\newcommand{\vang}[1]{\textcolor{yellow}{#1}}

%Định nghĩa màu cho khung và chữ viết, các bạn không nên xóa bỏ phần này

\setbeamercolor{vangxanh}{fg=blue,bg=yellow!20!white}
\setbeamercolor{vangden}{fg=black,bg=yellow!20!white}
\setbeamercolor{vangdo}{fg=red,bg=yellow!20!white}
\setbeamercolor{camden}{fg=black,bg=orange!20!white}
\setbeamercolor{camxanh}{fg=blue,bg=orange!20!white}
\setbeamercolor{bichdo}{fg=red,bg=cyan!15!white}
\setbeamercolor{bichden}{fg=black,bg=cyan!15!white}
\setbeamercolor{hongxanh}{fg=blue,bg=magenta!15!white}
\setbeamercolor{hongden}{fg=black,bg=magenta!15!white}
\setbeamercolor{lado}{fg=red,bg=lime!40}
\setbeamercolor{laden}{fg=black,bg=lime!40}
\setbeamercolor{laxanh}{fg=blue,bg=lime!40}

\title{\bf Reduce measures}
\author{\bf
\hong{Student: Phan Van Du\\
Advisor: Professor Laurent V\'eron}}
\institute{\duong{LABORATOIRE DE MATHEMATIQUES et  \\PHYSIQUE THEORIQUE\\
UNIVERSITE FRAN\c{C}OIS RABELAIS, TOURS--FRANCE}}
\date{\today}

\begin{document}

\frame{\titlepage}

\begin{frame}{\LARGE \bf  Contents}
\tableofcontents
\end{frame}


\section{$L^1$ theory}

\begin{frame}{$L^1$ theory}
\newtheorem*{pro}{The problem}
\begin{pro}
\[
\left\{
\begin{aligned}
-\Delta u + g(u) & = f && \text{in } \Omega , \\
u & = 0 && \text{on } \partial \Omega ,
\end{aligned}
\right.
\]
\end{pro}
\bigskip
\begin{beamercolorbox}{camxanh}
\begin{itemize}
\item $\Omega \subset \mathbb{R}^N$ bounded, open, smooth boundary
\item $g: \mathbb{R} \to \mathbb{R}$ continuous, non-decreasing, $g(0) = 0$
\item $f \in L^1(\Omega)$
\end{itemize}
\end{beamercolorbox}
\end{frame}

\begin{frame}{$L^1$ theory}
\begin{pro}
\[
\left\{
\begin{aligned}
-\Delta u + g(u) & = f && \text{in } \Omega , \\
u & = 0 && \text{on } \partial \Omega ,
\end{aligned}
\right.
\]
\end{pro}
\bigskip
\newtheorem*{weak}{Weak solution}
\begin{weak}
\[
\left\{
\begin{aligned}
& u \in L^1(\Omega ),\quad g(u) \in L^1(\Omega ) ,\\
&-\int_{\Omega} u \Delta \varphi + \int_{\Omega} g(u)\varphi = \int_{\Omega}f\varphi, \quad \forall \varphi \in C^2_0(\overline{\Omega}) .
\end{aligned}
\right.
\]
\end{weak}
\end{frame}

\begin{frame}{$L^1$ theory}
\begin{theorem}[Brezis-Strauss, 1973]
For $f \in L^1(\Omega)$, there is a unique solution to the problem
\[
\left\{
\begin{aligned}
-\Delta u + g(u) & = f && \text{in } \Omega , \\
u & = 0 && \text{on } \partial \Omega .
\end{aligned}
\right.
\]
\end{theorem}
\end{frame}

\section{Transitions to measure}

\begin{frame}{Transitions to measure}
\begin{pro}
\[
\left\{
\begin{aligned}
-\Delta u + g(u) & = \mu && \text{in } \Omega , \\
u & = 0 && \text{on } \partial \Omega ,
\end{aligned}
\right.
\]
\end{pro}
\bigskip
\begin{weak}
\[
\left\{
\begin{aligned}
& u \in L^1(\Omega ),\quad g(u) \in L^1(\Omega ) ,\\
&-\int_{\Omega} u \Delta \varphi + \int_{\Omega} g(u)\varphi = \int_{\Omega}\varphi\,d\mu, \quad \forall \varphi \in C^2_0(\overline{\Omega}) .
\end{aligned}
\right.
\]
\end{weak}
\end{frame}

\begin{frame}{Transition to measure}
\newtheorem*{cou}{Counterexample}
\begin{cou}
Assume $N\ge 3$. If $p \ge \frac{N}{N-2}$, then, for any $a\in \Omega$, the problem
\[
\left\{
\begin{aligned}
-\Delta u + |u|^{p-1}u & = \delta_a && \text{in } \Omega , \\
u & = 0 && \text{on } \partial \Omega ,
\end{aligned}
\right.
\]
has no solution $u\in L^p(\Omega)$.
\end{cou}
\bigskip
\begin{beamercolorbox}{camxanh}
\[
g(t) = |t|^{p-1}t
\]
\end{beamercolorbox}
\end{frame}

\begin{frame}{Transition to measure}
\begin{theorem}
Assume $N \ge 2$ and there exists $p \in [1, \frac{N}{N-2})$ such that
\[
|g(t)| \le C(|t|^p + 1), \quad\forall t\in \mathbb{R}.
\]
Then, for every $\mu\in \mathcal{M}(\Omega)$, there is a unique solution to the problem
\[
\left\{
\begin{aligned}
-\Delta u + g(u) & = \mu && \text{in } \Omega , \\
u & = 0 && \text{on } \partial \Omega ,
\end{aligned}
\right.
\]
\end{theorem}
\medskip
\begin{beamercolorbox}{camxanh}
\[
g(t) = |t|^{p-1}t \quad \Rightarrow \quad |g(t)| \le C(|t|^p + 1)
\]
\end{beamercolorbox}
\begin{beamercolorbox}{camxanh}
\[
g \text{ bounded} \quad \Rightarrow \quad |g(t)| \le C(|t|^p + 1)
\]
\end{beamercolorbox}
\end{frame}

\section{Reduce measures}

\begin{frame}{Reduce measures}
\begin{pro}
\[
\left\{
\begin{aligned}
-\Delta u + g(u) & = \mu && \text{in } \Omega , \\
u & = 0 && \text{on } \partial \Omega ,
\end{aligned}
\right.
\]
\end{pro}
\begin{beamercolorbox}{camxanh}
\begin{itemize}
\item $N \geq 2$
\item $g(t) = 0,\ \forall t \leq 0$
\end{itemize}
\end{beamercolorbox}
\medskip
\begin{beamercolorbox}{camxanh}
\[
g_n(t) = \min \{g(t), n \}\leq n,\quad\forall t
\]
\end{beamercolorbox}
\newtheorem*{thm}{}
\begin{thm}
\[
\left\{
\begin{aligned}
-\Delta u_n + g_n(u_n) & = \mu && \text{in } \Omega , \\
u_n & = 0 && \text{on } \partial \Omega ,
\end{aligned}
\right.
\]
\end{thm}
\end{frame}

\begin{frame}{Reduce measures}
\begin{thm}
\[
\left\{
\begin{aligned}
-\Delta u_n + g_n(u_n) & = \mu && \text{in } \Omega , \\
u_n & = 0 && \text{on } \partial \Omega ,
\end{aligned}
\right.
\]
\end{thm}
\bigskip
\begin{thm}
\begin{itemize}
\item $u_n \to u^* \quad\text{in }L^1(\Omega)$
\item $g(u_n) \to g(u^*) \quad\text{a.e in }\Omega$
\item $\Rightarrow \quad\text{Fatou}$
\end{itemize}
\end{thm}
\end{frame}

\begin{frame}{Reduce measures}
\begin{weak}
\[
\left\{
\begin{aligned}
& u \in L^1(\Omega ),\quad g(u) \in L^1(\Omega ) ,\\
&-\int_{\Omega} u \Delta \varphi + \int_{\Omega} g(u)\varphi = \int_{\Omega}\varphi\,d\mu, \quad \forall \varphi \in C^2_0(\overline{\Omega}) .
\end{aligned}
\right.
\]
\end{weak}
\bigskip
\newtheorem*{wss}{Subsolution}
\begin{wss}
\[
\left\{
\begin{aligned}
& u \in L^1(\Omega ),\quad g(u) \in L^1(\Omega ) ,\\
&-\int_{\Omega} u \Delta \varphi + \int_{\Omega} g(u)\varphi \leq \int_{\Omega}\varphi\,d\mu, \quad \forall \varphi \in C^2_0(\overline{\Omega}),\varphi \geq 0 .
\end{aligned}
\right.
\]
\end{wss}
\end{frame}

\begin{frame}{Reduce measures}
\begin{theorem}[1]
We have $u^*$ is the largest subsolution to our problem.
\end{theorem}
\bigskip
\begin{beamercolorbox}{camxanh}
\[-\int_{\Omega} u^* \Delta \varphi + \int_{\Omega} g(u^*)\varphi = \int_{\Omega}\varphi\,d\mu^*, \quad \forall \varphi \in C^2_0(\overline{\Omega})\]
\end{beamercolorbox}
\bigskip
\begin{theorem}[2]
We have $\mu^*$ is the largest good measure $\leq \mu$.
\end{theorem}
\end{frame}

\begin{frame}
\vspace{56pt}
\begin{center}
Thank you for your attention
\end{center}

\end{frame}

\end{document}
