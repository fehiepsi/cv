\documentclass[a4paper, 11pt]{report}
\usepackage[utf8]{inputenc}
\usepackage[english]{babel}
\usepackage{amsmath, amssymb, amsthm, graphicx, enumerate}
\usepackage[pdftex, bookmarks, bookmarksnumbered, pdfborder={0 0 0}, colorlinks=false]{hyperref}

\newtheorem{thm}{Theorem}[chapter]
\newtheorem{lmm}{Lemma}[chapter]
\newtheorem{crr}{Corollary}[chapter]
\newtheorem{prn}{Proposition}[chapter]
\theoremstyle{definition}\newtheorem*{rmk}{Remark}
\renewcommand{\thesection}{\arabic{section}.}
\DeclareMathOperator{\sgn}{sgn}
\DeclareMathOperator{\supp}{supp}
\DeclareMathOperator{\capa}{cap}
\DeclareMathOperator{\dive}{div}

\begin{document}
\pagenumbering{roman}

\begin{titlepage}
\hskip -3.1cm
\begin{tabular}[c]{lcr}
\includegraphics[width=4cm, height=1.1in]{logouniv.pdf} 
&\hskip 2cm
\includegraphics[width=4cm, height=1.1in]{logopoisson.png} 
&\hskip 2cm
\includegraphics[width=4.5cm, height=1.2in]{cnrslogo.pdf}
\end{tabular}

\begin{center}
\vspace{3cm}
\textbf{\Huge Master Thesis}

\vspace{2cm}
\textbf{\Huge Phan Van Du}

\vspace{2cm}
\textit{\Huge Reduced Measures}

\vspace{1cm}
\textit{\large arcording to H. Brezis, M. Marcus $\&$ A.C. Ponce}

\vspace{2cm}
\textbf{\Large Advisor:} \textit{\Large Laurent V\'eron}

\vspace{1cm}
\textbf{\Large Defense: Wednesday June $27^{th}$, 2012}
\end{center}
\end{titlepage}

\begin{abstract}
\mbox{}
\setcounter{page}{2}

\begin{flushright}
\textit{most written by Professor Laurent V\'eron}
\end{flushright}
Reduced measures have been introduced by Brezis, Marcus and Ponce in the study of semilinear elliptic equations in $\mathbb{R}^N$. Typical example is the following: consider the problem
\begin{equation}\label{abs1}
\left\{
\begin{aligned}
-\Delta u + g(u) &= \mu &&\text{in }\Omega,\\
u &= 0 && \text{on }\partial \Omega,
\end{aligned}
\right.
\end{equation}
where $g:\mathbb{R}\to \mathbb{R}$ is a continuous non-decreasing function with $g(t) = 0$ for all $t \leq 0$ and $\mu$ is a bounded Borel measure in $\Omega$. It is well-known that this problem cannot be solved for any measure if $g$ has a too strong growth at infinity. The idea of reduced measure comes from the approximation of \eqref{abs1} where $g$ is replaced by $g_k$ ($k>0$), where $g_k(r) = \min\{k, g(r)\}$ for all $r \in \mathbb{R}$. The approximate problem becomes
\[
\left\{
\begin{aligned}
-\Delta u_k + g_k(u_k) &= \mu &&\text{in }\Omega,\\
u_k &= 0 && \text{on }\partial \Omega.
\end{aligned}
\right.
\]
When $k$ increases, $u_k$ decreases and converges to some $u^*$, but $g_k (u_k)$ does not converge to $g(u^*)$. By Fatou’s Lemma, there is a loss, and finally, $u^*$ is a solution of a relaxed problem with a new measure $\mu^*$:
\[
\left\{
\begin{aligned}
-\Delta u + g(u) &= \mu^* &&\text{in }\Omega,\\
u &= 0 && \text{on }\partial \Omega.
\end{aligned}
\right.
\]
The measure $\mu^*$ is called the reduced measure. The properties of the reduced measures have been thoroughly studied by Brezis, Marcus and Ponce.

The subject of this Master thesis is to read the article of Brezis, Marcus and Ponce \cite{BMP} and to make a detailed review of their contents. The first chapter presents a result by Brezis and Strauss in 1973, which tells that \eqref{abs1} has a unique solution when $\mu \in L^1(\Omega)$. The second one makes a translation from $L^1$ data to measure data. Some preliminaries which are necessary for studying reduce measure will be displayed in the third chapter. Finally, Chapter 4 is devoted to the construction of measures and their properties.
\end{abstract}

\tableofcontents
\setcounter{page}{3}
\thispagestyle{empty}

\chapter{\texorpdfstring{$L^1$ theory}{L1 theory}}
\mbox{}
\pagenumbering{arabic}

Consider the problem:
\begin{equation}\label{main}
\left\{
\begin{aligned}
-\Delta u + g(u) & = f && \text{in } \Omega , \\
u & = 0 && \text{on } \partial \Omega ,
\end{aligned}
\right.
\end{equation}
with the following assumptions:
\begin{itemize}
\item $\Omega \subset \mathbb{R}^N$ is a bounded open set with smooth boundary.
\item $g: \mathbb{R} \to \mathbb{R}$ is a continuous, non-decreasing function such that $g(0) = 0$.
\end{itemize}

The main content of this chapter is the following theorem which is a result by Brezis-Strauss \cite[Theorem 1]{BS}:

\begin{thm}\label{L1thm}
For $f \in L^1(\Omega)$, the problem \eqref{main} has a unique weak solution $u$ such that
\begin{equation}\label{weak}
\left\{
\begin{aligned}
& u \in L^1(\Omega ),\quad g(u) \in L^1(\Omega ) ,\\
&-\int_{\Omega} u \Delta \varphi + \int_{\Omega} g(u)\varphi = \int_{\Omega}f\varphi, \quad \forall \varphi \in C^2_0(\overline{\Omega}) .
\end{aligned}
\right.
\end{equation}
\end{thm}

Here $C^2_0(\overline{\Omega}) = \{\varphi \in C^2(\overline{\Omega}): \varphi = 0 \text{ on } \partial \Omega\}$.

First, to prove this theorem, we will show the existence of a weak solution in case $f \in L^2(\Omega)$ and $g$ bounded.

\begin{lmm}\label{L2glmm}
For $f \in L^2(\Omega)$ and $g$ bounded, the problem \eqref{main} has a weak solution $u \in H^1_0(\Omega )$ such that
\[
\int_{\Omega} \nabla u . \nabla \varphi + \int_{\Omega} g(u)\varphi = \int_{\Omega}f\varphi, \quad \forall \varphi \in H^1_0(\Omega) .
\]
\end{lmm}

\begin{proof}
We prove this lemma by using the ``minimization'' method.

Considering the functional:
\begin{equation}\label{funJ}
J[u] = \frac{1}{2} \int_{\Omega}|\nabla u|^2 + \int_{\Omega}G(u) - \int_{\Omega}fu  \text{in } H^1_0(\Omega),
\end{equation}
where $G(t) = \int_0^t g(s)\,ds$.

Let $M = \sup |g|$. Due to $g(t)t \ge 0$, we have $G(t)\ge 0$ for all $t \in \mathbb{R}$. Moreover, by $u \in H^1_0(\Omega)$ and
\[
G(u) = \int_0^u g(s)\,ds \le M|u|,
\]
we have $G(u) \in L^2(\Omega)$. And since $\Omega$ is bounded, we have $\varphi \in L^1(\Omega)$. This means the functional $J[u]$ is well-defined in $H^1_0(\Omega)$.

Then, we will show that the functional $J[u]$ is bounded from below in $H^1_0(\Omega)$. Indeed, by using Cauchy's Inequality and then Poincar\'e's Inequality, we have
\begin{equation}\label{est1}
\int_{\Omega}fu \le \frac{1}{2\epsilon}\int_{\Omega}f^2 + \frac{\epsilon}{2}\int_{\Omega}u^2 \le \frac{1}{2\epsilon}\int_{\Omega}f^2 +  \frac{\epsilon C}{2} \int_{\Omega}|\nabla u|^2, \quad \forall \epsilon > 0,
\end{equation}
where $C>0$ is a constant depending only on $\Omega$. Choosing $\epsilon = C^{-1}$ and applying the above inequality to \eqref{funJ}, we get
\[
J[u] \ge - \frac{C}{2}\int_{\Omega}f^2, \quad \forall u \in H^1_0(\Omega).
\]

Next, put
\[
\inf J = \inf \{J[u]: u\in H^1_0(\Omega)\},
\]
and let $\{ u_n\}$ be a sequence in $H^1_0(\Omega)$ such that $\lim\limits_{n\to \infty} J[u_n] = \inf J$. It is clear that $\{J[u_n]\}$ is a bounded sequence. Note that in \eqref{est1}, if we choose $\epsilon = (2C)^{-1}$ and apply it to \eqref{funJ}, we will get the estimate:
\[
\int_{\Omega}|\nabla u|^2 \le 4J[u] + 4C\int_{\Omega}f^2.
\]

Due to the boundedness of the sequence $\{J[u_n]\}$, this estimate infers $\{u_n\}$ is a bounded sequence in $H^1_0(\Omega)$ (through the equivalent of the norm $\| \nabla u \|_{L^2}$ and the norm $\| u \|_{H^1}$ in $H^1_0(\Omega)$ when $\Omega$ is bounded).

By the reflexibility of $H^1_0(\Omega)$ space and the existence of a compact embedding from $H^1(\Omega)$ space to $L^2(\Omega)$ space, there exists $u \in H^1_0(\Omega )$ and a subsequence $\{u_{n_k}\}$ of $\{u_n\}$ such that
\begin{equation}\label{conv1}
u_{n_k} \to u\quad \text{in } L^2(\Omega),
\end{equation}
\begin{equation}\label{conv2}
u_{n_k} \to u \quad \text{a.e.\ in } \Omega,
\end{equation}
\begin{equation}\label{conv3}
\nabla u_{n_k} \rightharpoonup \nabla u \quad \text{weakly in } L^2(\Omega) .
\end{equation}

From \eqref{conv1}, we get
\begin{equation}\label{res1}
\int_{\Omega}fu_{n_k} \to \int_{\Omega}fu .
\end{equation}

From \eqref{conv2}, we get
\[
G(u_{n_k}) \to G(u)\quad \text{a.e.\ in }\Omega,
\]
and by Fatou's Lemma, we have
\begin{equation}\label{res2}
\liminf \int_{\Omega} G(u_{n_k}) \ge G(u).
\end{equation}

From \eqref{conv3}, we get
\[
\int_{\Omega}\nabla u_{n_k}.\nabla u \to \int_{\Omega}|\nabla u|^2  .
\]

Since
\[
\int_{\Omega} |\nabla u_{n_k}|^2 \ge 2\int_{\Omega} \nabla u_{n_k}.\nabla u - \int_{\Omega}|\nabla u|^2,
\]
we have
\begin{equation}\label{res3}
\liminf \int_{\Omega} |\nabla u_{n_k}|^2 \ge \int_{\Omega} |\nabla u|^2 .
\end{equation}

From \eqref{res1}, \eqref{res2} and \eqref{res3}, we conclude that
\[
\inf J = \liminf J[u_{n_k}] \ge J[u],
\]
which means that $u$ is a minimizer of the functional $J$.

Now we will show that this minimizer is indeed a solution to our problem.

For every $\epsilon \neq 0$ and $\varphi \in H^1_0(\Omega)$, we have
\[
\frac{J[u+\epsilon \varphi] - J[u]}{\epsilon} = \epsilon \left(\int_{\Omega}\nabla u . \nabla \varphi - \int_{\Omega} f\varphi \right) + \int_{\Omega}\frac{G(u+\epsilon \varphi) - G(\varphi)}{\epsilon} .
\]

So,
\begin{equation}\label{dev}
\lim_{\epsilon \to 0} \frac{J[u+\epsilon \varphi]-J[u]}{\epsilon} =\int_{\Omega}\nabla u . \nabla \varphi - \int_{\Omega} f\varphi + \int_{\Omega} g(u)\varphi .
\end{equation}

Moreover, we have 
\[
J[u+\epsilon \varphi] \ge J[u],\quad \forall \epsilon \in \mathbb{R},\forall\varphi \in H^1_0{(\Omega)} .
\]
 
Let $\epsilon \to 0^+$ and then $\epsilon \to 0^-$, we have
\[
\lim_{\epsilon \to 0} \frac{J[u+\epsilon \varphi]-J[u]}{\epsilon} = 0 .
\]

This and \eqref{dev} imply that $u$ is a solution to our problem.
\end{proof}

Removing the boundedness of $g$, we obtain

\begin{lmm}\label{L2lmm}
For $f \in L^2(\Omega)$, the problem \eqref{main} has a weak solution $u\in H^1_0(\Omega)$ such that $g(u) \in L^1(\Omega )$ and
\[
\int_{\Omega} \nabla u . \nabla \varphi + \int_{\Omega} g(u)\varphi = \int_{\Omega}f\varphi, \quad \forall \varphi \in H^1_0(\Omega)\cap L^{\infty}(\Omega).
\]
\end{lmm}

We will use Vitali's ``compactness'' method to prove this lemma.

\begin{lmm}[Vitali's Theorem]\label{vlmm}
Let $\{f_n\}$ be a sequence in $L^p(\Omega)$ with $1 \le p < \infty$. Assume that
\begin{enumerate}[i)]
\item $\forall \epsilon > 0, \exists \delta > 0$ such that for every $n$ and every measurable set $A \subset \Omega$ with $|A| < \delta$, we have $\int_{A}|f_n| < \epsilon$.
\item $f_n \to f \quad\text{a.e.\ in }\Omega$.
\end{enumerate}
Then, $f \in L^p(\Omega)$ and $f_n \to f$ in $L^p(\Omega)$.
\end{lmm}

\begin{proof}[Proof of Lemma \ref{L2lmm}]
\mbox{}

Let
\[
g_n(t)=\left\{
\begin{aligned}
& g(t) \quad &&\text{if } |t| \le n,\\
& g(n) \quad &&\text{if } t > n,\\
& -g(n) \quad &&\text{if } t < -n.
\end{aligned}
\right.
\]

By Lemma \ref{L2glmm}, for each $g_n$, there is $u_n \in H^1_0(\Omega)$ such that
\begin{equation}\label{sol3}
\int_{\Omega} \nabla u_n . \nabla \varphi + \int_{\Omega} g_n(u_n)\varphi = \int_{\Omega}f\varphi \quad \forall \varphi \in H^1_0(\Omega).
\end{equation}

Replacing $\varphi$ by $u_n$, we get
\begin{equation}\label{equa1}
\int_{\Omega} |\nabla u_{n}|^2 + \int_{\Omega} g_n(u_n)u_n = \int_{\Omega}fu_n,
\end{equation}
which implies that
\[
\| \nabla u_n\|^2_{2} \le \int_{\Omega}fu_n .
\]

Using H\"older's Inequality and Poincar\'e's Inequality, we get the estimate:
\begin{equation}\label{ineq1}
\int_{\Omega}fu_n \le \| f\|_{2}\| u_n\|_{2} \le C \| f\|_{2}\| \nabla u_n\|_2,
\end{equation}
where $C>0$ is a constant depending only on $\Omega$. 

This leads to
\begin{equation}\label{ineq2}
\| \nabla u_n\|_{2} \le C \| f\|_{2},
\end{equation}
which means that $\{u_n\}$ is a bounded sequence in $H^1_0(\Omega)$.

So there exist $u \in H^1_0(\Omega )$ and a subsequence $\{u_{n_k}\}$ of $\{u_n\}$ such that
\[
u_{n_k} \to u\quad \text{in } L^2(\Omega),
\]
\[
u_{n_k} \to u \quad \text{a.e.\ in } \Omega,
\]
\begin{equation}\label{conv6}
\nabla u_{n_k} \rightharpoonup \nabla u \quad \text{weakly in } L^2(\Omega) .
\end{equation}

For each $x\in \Omega$ such that $u_{n_k}(x) \to u(x)$, we choose $M_x$ large enough such that $n_k > |u(x)| + 1$ and $|u_{n_k}(x) - u(x)| < 1$ for each $k > M_x$. These inequalities imply that $n_k > |u_{n_k}(x)|$ for each $k > M_x$. So, by the definition of sequence $g_n$, we get 
\[
g_{n_k}(u_{n_k}(x)) = g(u_{n_k}(x)), \quad\forall k > M_x,
\]
which implies
\[
g_{n_k}(u_{n_k})\to g(u)\quad \text{a.e.\ in }\Omega .
\]

Now, we will prove that $g_{n_k}(u_{n_k})$ is equi-integrable.

First, from \eqref{equa1}, we have
\[
\int_{\Omega} g_{n_k}(u_{n_k})u_{n_k} \le \int_{\Omega} f u_{n_k}.
\]

Then, by \eqref{ineq1} and \eqref{ineq2}, we get
\[
\int_{\Omega} g_{n_k}(u_{n_k})u_{n_k} \le C^2\| f\|^2_{2}.
\]

For each $M > 0$ and for each measurable set $E$ in $\Omega$, we have
\begin{align*}
\int_E |g_{n_k}(u_{n_k})|& \le \int_{E\cap [|u_{n_k}| \le M]} |g_{n_k}(u_{n_k})| + \int_{E\cap[|u_{n_k}| > M]} |g_{n_k}(u_{n_k})|\\
&\le \max_{[-M,M]} |g|.|E| + \frac{1}{M}\int_{E\cap[|u_{n_k}| > M]} g_{n_k}(u_{n_k})u_{n_k}\\
&\le \max_{[-M,M]} |g|.|E| + \frac{C^2\| f\|^2_2}{M}.
\end{align*}

For $\epsilon > 0$, let $M$ large enough such that
\[
\frac{C^2\| f\|^2_{2}}{M} < \frac{\epsilon}{2},
\]
and with this fixed $M$, let $\delta > 0$ small enough such that
\[
\max_{[-M,M]} |g|.|E| < \frac{\epsilon}{2}\quad \text{if } |E| < \delta.
\]

So we have
\[
\int_E|g_{n_k}(u_{n_k})| < \epsilon \quad \text{if } |E|<\delta ,
\]
meaning that $\{g_{n_k}(u_{n_k})\}$ is equi-integrable.

Using Vitali's Theorem (Lemma \ref{vlmm}), we conclude $g(u)\in L^1(\Omega)$ and
\begin{equation}\label{conv7}
g_{n_k}(u_{n_k})\to g(u)\quad \text{in } L^1(\Omega).
\end{equation}

From \eqref{conv6} and \eqref{conv7}, for every $\varphi \in H^1_0(\Omega)\cap L^{\infty}(\Omega)$, we have
\[
\int_{\Omega}\nabla u_{n_k} . \nabla \varphi + \int_{\Omega}g(u_{n_k})\varphi \to \int_{\Omega}\nabla u . \nabla \varphi + \int_{\Omega}g(u)\varphi.
\]

By \eqref{sol3}, $u$ is a solution to our problem.
\end{proof}

\begin{rmk}
In the proofs of the two above lemmas, we do not need the monotone of $g$. This means that we can get the same results for continuous function $g$ such that $g(t)t \ge 0$ for all $ t\in \mathbb{R}$.
\end{rmk}

For the transition from $f\in L^2(\Omega)$ to $f\in L^1(\Omega)$, we need the following a priori estimates.

\begin{lmm}\label{L1e1lmm}
For $f \in L^1(\Omega)$, if
\begin{equation}\label{wmp1}
-\int_{\Omega} f\Delta\varphi \le 0, \quad \forall \varphi \in C_0^2(\overline{\Omega}), \varphi \ge 0,
\end{equation}
then
\[
f \le 0 \quad \text{a.e.\ in }\Omega .
\]
\end{lmm}
\begin{proof}
\mbox{}

For each $\psi \in C_c^{\infty}(\Omega)$, there exists $\varphi \in C^2_0(\overline{\Omega})$ such that $-\Delta \varphi = \psi$ in $\Omega$.

By classical maximum principle, if $\psi \ge 0$, then $\varphi \ge 0$. 

So if $f$ satisfies \eqref{wmp1}, then
\[
\int_{\Omega} f\psi \le 0 \quad \forall \psi \in C_c^{\infty}(\Omega), \psi \ge 0.
\]

This implies
\[
f \le 0 \quad \text{a.e.\ in } \Omega. \qedhere
\]
\end{proof}

\begin{lmm}\label{L1e2lmm}
For $f \in L^1(\Omega)$ and $u\in L^1(\Omega)$ such that
\[
-\int_{\Omega} u\Delta \varphi = \int_{\Omega} f \varphi, \quad \forall \varphi \in C^2_0(\overline{\Omega}),
\]
we have
\[
\| u \|_{L^1(\Omega)} \le C\| f\|_{L^1(\Omega)},
\]
where $C$ is a constant depending only on $\Omega$.
\end{lmm}
\begin{proof}
\mbox{}

First, we suppose that $u\in C^2_0(\overline{\Omega})$.

By Lemma \ref{L2glmm}, let $\varphi \in H^1_0(\Omega)$ is a solution to the problem:
\begin{equation}\label{rep}
\int_{\Omega} \nabla \varphi.\nabla \psi = \int_{\Omega} \sgn(u) \psi, \quad \forall \psi \in H^1_0(\Omega).
\end{equation}

Choose $\overline{\varphi} \in C^2_0(\overline{\Omega})$ such that $-\Delta \overline{\varphi} = 1$ in $\Omega$.

Since $\sgn(u)\le 1$, we get
\[
-\int_{\Omega} (\varphi - \overline{\varphi})\Delta \psi \le 0, \quad \forall \psi \in C_0^2(\overline{\Omega}), \psi \ge 0,
\]
which implies (by Lemma \ref{L1e1lmm})
\begin{equation}\label{max1}
\varphi \le \overline{\varphi} \quad \text{a.e.\ in }\Omega.
\end{equation}

Similarly, since $-\sgn(u)\le 1$, we get
\[
-\int_{\Omega} (-\varphi - \overline{\varphi})\Delta \psi \le 0, \quad \forall \psi \in C_0^2(\overline{\Omega}), \psi \ge 0,
\]
which implies (by Lemma \ref{L1e1lmm})
\begin{equation}\label{max2}
-\varphi \le \overline{\varphi} \quad \text{a.e.\ in }\Omega.
\end{equation}

From \eqref{max1} and \eqref{max2}, we get
\[
\| \varphi\|_{\infty} \le C,
\]
where $C = \| \overline{\varphi}\|_{\infty}$ is a constant depending only on $\Omega$.

Replace $\psi$ by $u$ in \eqref{rep}, we get
\[
\int_{\Omega}\sgn(u)u = \int_{\Omega}\nabla u.\nabla \varphi = \int_{\Omega}\Delta u .\varphi,
\]
and thus,
\begin{equation}\label{case1}
\| u\|_{1} \le \| \Delta u\|_{1} \| \varphi\|_{\infty} \le C\| \Delta u\|_{1}.
\end{equation}

Now, we consider $u\in L^1(\Omega)$. Let $\{f_n\}$ be a sequence in $C^{\infty}_c(\Omega)$ which converges to $f$ in $L^1(\Omega)$. For each $f_n$, choose $u_n \in C^2_0(\overline{\Omega})$ such that
\begin{equation}\label{dddd}
\Delta u_n = f_n .
\end{equation}

By \eqref{case1}, we have
\[
\| u_m - u_n \|_1 \le C \| f_m - f_n \|_1 ,\quad \forall m, n,
\]
which implies that $\{u_n\}$ is a Cauchy sequence in $L^1(\Omega)$. So, $\{ u_n\}$ converges to $v \in L^1(\Omega)$.

From \eqref{dddd}, we have
\[
-\int_{\Omega} u_n\Delta \varphi = \int_{\Omega} f_n \varphi, \quad \forall \varphi \in C^2_0(\overline{\Omega}).
\]

Taking the limits of both sides gives
\[
-\int_{\Omega} v\Delta \varphi = \int_{\Omega} f \varphi, \quad \forall \varphi \in C^2_0(\overline{\Omega}),
\]
which implies
\[
\int_{\Omega} (u-v)\Delta \varphi = 0, \quad \forall \varphi \in C^2_0(\overline{\Omega}).
\]

Since
\[
\{ \Delta \varphi: \varphi \in C_0^2(\overline{\Omega})\} \subset C_c^{\infty}(\Omega),
\]
we have
\[
u = v\quad \text{a.e.\ in } \Omega .
\]

From \eqref{case1}, we obtain
\[
\| u_n\|_1 \le C\| f_n\|_1.
\]

Taking the limits of both sides, we get the conclusion.
\end{proof}

\begin{lmm}\label{L1e3lmm}
For $f \in L^1(\Omega)$ and $u\in L^1(\Omega)$ such that
\[
-\int_{\Omega} u\Delta \varphi = \int_{\Omega} f \varphi, \quad \forall \varphi \in C^2_0(\overline{\Omega}).
\]
Then
\[
\int_{\Omega} f.\sgn(u) \ge 0 .
\]
\end{lmm}

\begin{proof}
\mbox{}

Let $p\in C^2(\mathbb{R})$ such that 
\begin{equation}\label{1}
p(0)=0,\quad p' \ge 0,\quad\text{and}\quad |p|\le 1 .
\end{equation}

Let $\{f_n\}$ be a sequence in $C^{\infty}_c(\Omega)$ which converges to $f$ in $L^1(\Omega)$. For each $f_n$, choose $u_n \in C^2_0(\overline{\Omega})$ such that $\Delta u_n = f_n$,
which implies
\begin{equation}\label{dd3}
\int_{\Omega}\nabla u_n.\nabla \varphi = \int_{\Omega} f_n \varphi ,\quad \forall \varphi \in C^2_0(\overline{\Omega}).
\end{equation}

For each $n$, from \eqref{1}, we have $p(u_n)\in C^2_0(\overline{\Omega})$. So, in \eqref{dd3}, if we take $\varphi = p(u_n)$, then
\begin{equation}\label{dd2}
\int_{\Omega} f_n.p(u_n) = \int_{\Omega} \nabla u_n.\nabla(p(u_n)) = \int_{\Omega} p'(u_n)|\nabla u_n|^2 \ge 0.
\end{equation}

By the same argument as in Lemma \ref{L1e2lmm}, we obtain
\[
u_n \to u \quad \text{in }L^1(\Omega).
\]

We may take a subsequence $\{n_k \}$ and $h \in L^1(\Omega)$ such that
\[
u_{n_k} \to u \quad \text{a.e.\ in }\Omega,
\]
\[
f_{n_k} \to f \quad \text{a.e.\ in }\Omega,
\]
\[
|f_{n_k}| < h \quad \text{a.e.\ in } \Omega,\forall k.
\]

Consequently,
\[
f_{n_k}.p(u_{n_k}) \to f.p (u)\quad \text{a.e.\ in } \Omega,
\]
\[
|f_{n_k}.p(u_{n_k})| < h1 \quad \text{a.e.\ in } \Omega,\forall k.
\]

The dominated convergence theorem gives
\[
\int_{\Omega} f_{n_k}.p(u_{n_k}) \to \int_{\Omega} f.p(u),
\]
which together with \eqref{dd2} imply
\begin{equation}\label{dd1}
\int_{\Omega} f.p(u) \ge 0.
\end{equation}

Now let $\{p_n\}$ be a sequence of non-decreasing functions in $C^2(\mathbb{R})$ such that $p_n(0) = 0$, $p_n(t) = 1$ if $t > \frac{1}{n}$, $p_n(t) = -1$ if $t < -\frac{1}{n}$.

From \eqref{dd1}, we have
\[
\int_{\Omega} f.p_n(u) \ge 0, \quad \forall n.
\]

Moreover, it is clear that
\[
p_n(u)\to \sgn(u),
\]
which implies
\[
f.p_n(u)\to f.\sgn(u).
\]

Since $|p_n(u)|\le 1$ for all $n$, applyng the dominated convergence theorem, we reach the conclusion.
\end{proof}

Now we prove Theorem \ref{L1thm}.

\begin{proof}[Proof of Theorem \ref{L1thm}]
\mbox{}

Let $\{f_n\}$ be a sequence in $C^{\infty}_c(\Omega)$ which converges to $f$ in $L^1(\Omega)$.

By Lemma \ref{L2lmm}, for each $f_n$, there exists $u_n\in H^1_0(\Omega)$ such that $g(u_n)\in L^1(\Omega)$ and
\begin{equation}\label{ddd4}
-\int_{\Omega} u_n\Delta \varphi +\int_{\Omega} g(u_n)\varphi = \int_{\Omega} f_n \varphi, \quad\forall \varphi\in C^2_0(\overline{\Omega}).
\end{equation}

So, for each $m,n$, we have
\[
-\int_{\Omega} (u_m-u_n)\Delta \varphi = \int_{\Omega} (f_m-f_n-g(u_m)+g(u_n)) \varphi, \quad\forall \varphi\in C^2_0(\overline{\Omega}).
\]

By Lemma \ref{L1e3lmm}, we have
\[
\int_{\Omega} (f_m-f_n-g(u_m)+g(u_n)).\sgn(u_m-u_n) \ge 0, \quad \forall m,n.
\]

Because $g$ is non-decreasing, for each $m,n$,
\begin{align*}
\| g(u_m)-g(u_n)\|_1 &= \int_{\Omega} (g(u_m)-g(u_n)).\sgn(u_m-u_n)\\
&\le \int_{\Omega} (f_m-f_n).\sgn(u_m-u_n),
\end{align*}
which gives us
\begin{equation}\label{ddd0}
\| g(u_m)-g(u_n)\|_1\le \| f_m - f_n \|_1 \quad \forall m,n.
\end{equation}

In addition, Lemma \ref{L1e2lmm} give us
\begin{equation}\label{ddd1}
\| u_m - u_n\|_1 \le \| f_m-f_n-g(u_m)+g(u_n) \|_1 \le 2\| f_m - f_n \|_1\quad \forall m,n.
\end{equation}

By \eqref{ddd1}, we conclude that $\{u_n\}$ is a Cauchy sequence in $L^1(\Omega)$. So, $\{u_n\}$ converges to $u$ in $L^1(\Omega)$. We take a subsequence $\{u_{n_k}\}$ which converges to $u$ a.e.\ on $\Omega$. By the continuity of $g$, we have 
\begin{equation}\label{ddd2}
g(u_{n_k})\to g(u) \quad \text{a.e.\ on }\Omega.
\end{equation}

The inequality \eqref{ddd0} implies $\{g(u_n)\}$ is a Cauchy sequence in $L^1(\Omega)$. So, together with \eqref{ddd2}, we conclude that $g(u_n)$ converges to $g(u)$ in $L^1(\Omega)$.

Taking the limits of both sides in \eqref{ddd4}, we obtain that $u$ is a solution to our problem.

Now to prove the uniqueness, suppose that $v$ is a solution to our problem. We have
\begin{equation}\label{ddd6}
-\int_{\Omega}(u-v)\Delta \varphi = \int_{\Omega}(g(v)-g(u))\varphi, \quad \forall \varphi\in C^2_0(\overline{\Omega}).
\end{equation}

From Lemma \ref{L1e3lmm}, we have
\[
\int_{\Omega}(g(v)-g(u))\sgn(u-v) \ge 0.
\]

Since $g$ is non-decreasing, we get
\[
(g(v)-g(u))\sgn(u-v) \le 0.
\]

So we have
\[
(g(v)-g(u))\sgn(u-v) = 0\quad \text{a.e.\ on }\Omega,
\]
which implies
\[
g(u) = g(v) \quad \text{a.e.\ on }\Omega.
\]

This and \eqref{ddd6} give
\[
\int_{\Omega}(u-v)\Delta \varphi = 0, \quad \forall \varphi\in C^2_0(\overline{\Omega}).
\]

Since
\[
\{\Delta \varphi:\varphi\in C^2_0(\overline{\Omega})\} \subset C_c^{\infty}(\Omega),
\]
we have
\[
u = v \quad \text{a.e.\ on }\Omega,
\]
which solves the uniqueness of solutions to our problem.
\end{proof}

\chapter{Transition to measures}

\section{Definition of measures}
\mbox{}

Recall that because of the compactness of $\overline{\Omega}$, a Radon measure on $\overline{\Omega}$ is a bounded Borel measure. By Riesz Representation Theorem, it can be considered as a continuous linear functional on $C(\overline{\Omega})$, through the isometric isomorphism:
\[
\mu \mapsto I_{\mu},
\]
where
\[
I_{\mu}: f\mapsto \langle \mu, f\rangle = \int_{\overline{\Omega}}f\,d\mu .
\]

Denote $C_0(\overline{\Omega}) = \{u\in C(\overline{\Omega}): u = 0 \text{ on }\partial\Omega\}$.

From now, by a \emph{measure} $\mu$ we mean a continuous linear functional on $C_0(\overline{\Omega})$. It can be considered as a Radon measure on $\overline{\Omega}$ such that $|\mu | (\partial \Omega) = 0$, as showed in the following theorem:

\begin{thm}\label{meathm}
Given $\mu\in [C_0(\overline{\Omega})]^*$, there exists a unique $[\tilde{\mu} \in C(\overline{\Omega})]^*$ such that
\begin{equation}\label{2333}
\tilde{\mu} = \mu \quad \text{on } C_0(\overline{\Omega}),
\end{equation}
and
\begin{equation}\label{2334}
|\tilde{\mu}|(\partial \Omega) = 0.
\end{equation}
In addition, the map $\mu \mapsto \tilde{\mu}$ is a linear isometry.
\end{thm}

To prove this theorem, we need the following lemma:
\begin{lmm}\label{mealmm}
Given $\epsilon > 0$, there exists $\delta > 0 $ such that if
\[
f \in C_0(\overline{\Omega}),\quad |f| \le 1 \text{ in } \overline{\Omega},\quad\text{and}\quad\supp f \subset \overline{\Omega}\setminus \Omega_{\delta},
\]
then
\[
|\langle \mu ,f \rangle |\le \epsilon.
\]
\end{lmm}

Here $\Omega_{\delta} = \{ x\in \Omega : d(x,\partial \Omega) > \delta\}$.

\begin{proof}
We prove this lemma by contradiction.

Assume that there exists $\epsilon > 0$ and a sequence $\{f_n\}\subset C_0(\overline{\Omega})$ such that for each $n$,
\[
\| f_n \|_{\infty} \le 1 \text{ in } \overline{\Omega},\quad\supp f_n \subset \overline{\Omega}\setminus \Omega_{1/n},
\]
and
\[
|\langle \mu ,f_n \rangle |> \epsilon.
\]

Due to the continuity of $\mu$, we can choose $\delta > 0$ such that for every $f, g \in C(\overline{\Omega})$, if 
\[
\| f - g \|_{\infty} \le \delta , 
\] 
then 
\[
|\langle \mu, f -g \rangle |< \epsilon / 2 .
\]

For each $n$, $f_n$ is uniformly continuous on the compact set $\overline{\Omega}\setminus \Omega_{1/n}$.

As a result, we can choose $m > n$ such that for every $x, y \in \overline{\Omega}$, if
\[
|x - y| \le 1/m,
\]
then
\[
|f_n(x) - f_n(y)| < \delta .
\]

This implies that
\[
\sup_{\overline{\Omega}\setminus \Omega_{1/m}} |f_n| \le \delta .
\]

By Urysohn's Lemma, we can choose $g_n \in C(\overline{\Omega})$ such that
\[
\supp g_n \subset \Omega,\quad 0 \le g_n.\sgn (f_n) \le |f_n|,
\]
and
\[
g_n = f_n \quad\text{in }\overline{\Omega}_{1/m}.
\]

Having these properties, the sequence $\{g_n\} \subset C_0(\overline{\Omega})$ satisfies that for each $n$,
\[
\| g_n\|_{\infty} \le 1 \text{ in } \overline{\Omega},\quad\supp g_n \subset \Omega\setminus \Omega_{1/n} ,
\]
and
\[
|\langle \mu, g_n\rangle | > \epsilon / 2 ,
\]
which together imply that there is a subsequence $\{g_{n_k}\}$ such that all $\supp g_{n_k}$ are disjoint.

For any $k \ge 1$, set $G_k = \sum^{k}_{j=1} g_{n_j}$. We have 
\[
\| G_k \|_{\infty} \le 1,\quad \supp G_k \subset \Omega,
\]
and
\[
|\mu, G_k| > k\epsilon / 2,
\]
which imply that
\[
\| \mu \|_{[C_0(\overline{\Omega})]^*} > k\epsilon / 2,\quad \forall k.
\]

This gives us a contradiction.
\end{proof}

\begin{rmk}
We can replace the condition $|f| \le 1$ in the above lemma by $|f| \le C$, where $C > 0$ is a given constant.
\end{rmk}

\begin{proof}[Proof of Theorem \ref{meathm}]
\mbox{}

For $f \in C(\overline{\Omega})$, by Urysohn's Lemma, we can construct a sequence $\{ f_n \} \subset C_0(\overline{\Omega})$ such that
\begin{equation}\label{2111}
0 \le f_n.\sgn (f) \le |f|\quad \text{and} \quad f_n = f \text{ in } \Omega_{1/n}.
\end{equation}

For each $m > n$, by \eqref{2111}, we have
\[
\| f_m - f_n \|_{\infty} \le \| f\|_{\infty}\quad \text{and}\quad f_m = f_n \text{ in } \Omega_{1/n},
\]
which also imply that
\[
\supp (f_m - f_n) \subset \overline{\Omega}\setminus \Omega_{1/n}.
\]

For $\epsilon > 0$, we can choose $M > 0$ large enough such that $M^{-1}<\delta$, with $\delta$ comes from Lemma \ref{mealmm}. 

For all $m > n \ge M$, we have
\[
|\langle \mu ,f_m - f_n \rangle |\le \epsilon.
\]


Thus, $\{\langle \mu, f_n\rangle\}$ is a Cauchy sequence in $\mathbb{R}$, and so it converges to some limit.

Moreover, also by Lemma 2.1, this limit is independent of the way we construct a sequence $\{ f_n\}$. Indeed, if $\{ g_n\}$ is a subsequence in $C_0(\overline{\Omega})$ which satisfies \eqref{2111}, then for $\epsilon > 0$, choose $M > 0$ as above, we will get
\[
|\langle \mu ,f_n - g_n \rangle |\le \epsilon, \quad \forall n \ge M,
\]
which implies $\langle \mu ,f_n \rangle $ and $\langle \mu ,g_n \rangle $ have the same limit. 

So it is safe to set $\langle \tilde{\mu}, f\rangle$ as this limit.

Because $\mu$ is linear, it is clear that $\tilde{\mu}$ is a linear functional on $C(\overline{\Omega})$. Its continuity is given by the estimate:
\begin{align*}
\langle \tilde{\mu} , f \rangle = \lim_{n \to \infty} \langle \mu, f_n \rangle &\le \liminf_{n\to\infty}\| \mu \|_{[C_0(\overline{\Omega})]^*} \| f_n \|_{\infty} \\
&\le \| \mu \|_{[C_0(\overline{\Omega})]^*} \| f \|_{\infty} \quad \text{(by \eqref{2111})}.
\end{align*}

We have proved that $\tilde{\mu} \in [C(\overline{\Omega})]^*$ and it satisfies \eqref{2333}.

Now we show that it satisfies \eqref{2334}.

For $\epsilon > 0$, we have $f \in C(\overline{\Omega})$ such that
\begin{equation}\label{h1}
\| f\|_{\infty}\le 1\quad \text{and}\quad \langle \tilde{\mu}, f \rangle > \| \tilde{\mu}\|_{[C(\overline{\Omega})]^*} - \epsilon .
\end{equation}

Choosing a sequence $\{f_n\} \subset C_0(\overline{\Omega})$ which satisfies \eqref{2111}, we have
\[
\langle \tilde{\mu} , f \rangle = \lim_{n \to \infty} \langle \mu, f_n \rangle.
\]

And then, for $n$ large enough, we have
\begin{equation}\label{h2}
|\langle \mu, f_n - f \rangle | \le \epsilon / 3.
\end{equation}

By the same argument as in the proof of Lemma 2.1, we obtain $g\in C_0(\overline{\Omega})$ is a function such that
\begin{equation}\label{h3}
\| g\|_{\infty}\le 1,\quad \supp g \subset \Omega, \quad \text{and}\quad |\langle \mu, g - f_n \rangle | < \epsilon / 3.
\end{equation}

From \eqref{h1}, \eqref{h2}, and \eqref{h3}, we conclude that there exists $g\in C_0(\overline{\Omega})$ such that
\[
\| g\|_{\infty}\le 1 ,\quad \supp g \subset \Omega,\quad \text{and} \quad|\langle \tilde{\mu} , g \rangle | > \| \tilde{\mu}\|_{[C(\overline{\Omega})]^*} - \epsilon / 3.
\]

Let $m$ large enough such that $\supp g \subset \Omega_{1/m}$. Then
\[
|\tilde{\mu} | (\overline{\Omega}_{1/m})\ge \int_{\overline{\Omega}_{1/n}} g \,d\tilde{\mu} = \int_{\overline{\Omega}} g \,d\tilde{\mu} > |\tilde{\mu} | (\overline{\Omega}) -\epsilon / 3 .
\]

Consequently,
\[
|\tilde{\mu} | (\overline{\Omega}\setminus \overline{\Omega}_{1/m}) < \epsilon / 3,
\]
which implies that
\[
|\tilde{\mu} | (\overline{\Omega}\setminus \Omega) < \epsilon / 3 \quad \forall \epsilon > 0.
\]

Hence $|\tilde{\mu} | (\partial \Omega ) = 0$, which is \eqref{2334}.

Now, we prove that 
\[
\| \tilde{\mu}\|_{[C(\overline{\Omega})]^*} = \| \mu\|_{[C_0(\overline{\Omega})]^*}.
\]

By \eqref{2333}, it is clear that 
\[
\| \tilde{\mu}\|_{[C(\overline{\Omega})]^*} \ge \| \mu\|_{[C_0(\overline{\Omega})]^*}.
\]

For the reverse side, let $f\in C(\overline{\Omega})$. There exists a sequence $\{f_n\}\subset C_0(\overline{\Omega})$ which satisfies \eqref{2111}, and from
\[
\langle \tilde{\mu} , f \rangle = \lim_{n \to \infty} \langle \mu, f_n \rangle \le \| \mu\|_{[C_0(\overline{\Omega})]^*}\| f\|_{\infty},
\]
we get
\[
\| \tilde{\mu}\|_{[C(\overline{\Omega})]^*} \le \| \mu\|_{[C_0(\overline{\Omega})]^*}.
\]

Next, we prove the uniqueness of $\tilde{\mu}$.

Assume that there are $\mu_1, \mu_2$ both satisfy \eqref{2333} and \eqref{2334}.

For $f \in C(\overline{\Omega})$, we construct a sequence $\{f_n\}$ in $C_0(\overline{\Omega})$ which satisfies \eqref{2111}.

We have
\[
\| f_n \|_{\infty} \le \| f\|_{\infty},
\]
and
\[
f_n \to f \quad \text{in }\Omega.
\]

So, by the dominated convergence theorem, we get
\[
\int_{\Omega}f_n \,d\mu_i \to \int_{\Omega}f \, d\mu_i,\quad \text{for }i=1,2.
\]

Furthermore, by \eqref{2333}, we have
\[
\int_{\Omega}f_n \,d\mu_1 = \int_{\Omega}f_n \, d\mu_2\quad, \forall n,
\]
which leads to
\[
\int_{\Omega}f \,d\mu_1 = \int_{\Omega}f \, d\mu_2.
\]

Moreover, because both $\mu_1$ and $ \mu_2$ satisfy \eqref{2334}, we conclude that
\[
\int_{\overline{\Omega}}f \,d\mu_1 = \int_{\overline{\Omega}}f \, d\mu_2.
\]

And thus, $\mu_1 =\mu_2$.

Finally, the map $\mu \mapsto \tilde{\mu}$ is linear due to the way we construct $\tilde{\mu}$.
\end{proof}

We denote by $\mathcal{M}(\Omega)$ the space of such measures, which are Radon measures on $\overline{\Omega}$ vanishing on the boundary of $\Omega$. From the above theorem, its norm is given by
\[
\| \mu \|_{\mathcal{M}} = \sup \left\{\int_\Omega \varphi \, d\mu : \varphi \in C_0(\overline{\Omega}) \text{ and } \|\varphi\|_{\infty} \le 1\right\}.
\]

The following proposition is useful in the sequel. It is clear due to the density of $C_c^{\infty}(\Omega)$ space in $C_0(\overline{\Omega})$ space (with norm $L^{\infty}(\Omega)$).

\begin{prn}\label{21prn}
Let $\mu, \nu \in \mathcal{M}(\Omega)$. We have $\mu \leq \nu$ in $\Omega$ if and only if $\mu \leq \nu$ in $\mathcal{D}'(\Omega)$, which is
\[
\int_{\Omega}\varphi\,d\mu \leq \int_{\Omega}\varphi\,d\nu,\quad\forall \varphi \in C_c^{\infty}(\Omega),\varphi \geq 0.
\]
\end{prn}

Here by $\mu \le \nu$ (in $\Omega$) we mean that for every Borel subset E of $\Omega$,
\[
\mu(E) \le \nu(E).
\]

\begin{rmk}
We consider $L^1(\Omega)$ as a subspace of $\mathcal{M}(\Omega)$ by the embedding:
\begin{equation}\label{L12M}
f \mapsto T_f,
\end{equation}
where 
\[
\langle T_f, \varphi \rangle = \int_{\Omega}f\varphi, \quad\forall \varphi \in C_0(\overline{\Omega}).
\]
Moreover, we have
\[
\|T_f\|_{\mathcal{M}} = \| f \|_1, \quad \forall f \in L^1(\Omega).
\]
\end{rmk}

\section{Convolution with measures}
\mbox{}

Let $\mu \in \mathcal{M}(\Omega)$. Here we will demonstrate one way, through \emph{convolution}, to approximate $\mu$ by functions in $C(\overline{\Omega})$.

Recall that a sequence of mollifiers $\{\rho_n\}$ is any sequence of functions in $C_c^{\infty}(\mathbb{R}^N)$ such that
\begin{equation}\label{mol}
\supp \rho_n \subset \overline{B(0, 1/n)},\quad\int_{\mathbb{R}^N}\rho_n = 1 ,\quad\text{and}\quad \rho_n \ge 0 \text{ on } \mathbb{R}^N.
\end{equation}

Here is a well-known example of such sequence of mollifiers. First let $\rho$ is the function:
\[
\rho(x) = \left\{
\begin{aligned}
&e^{1/(x^2 -1)}&&\text{if } |x|< 1 , \\
& 0&&\text{if } |x|\ge 1.
\end{aligned}
\right.
\]
Then let $C = 1/\int_{\mathbb{R}^N}\rho$, and put 
\[
\rho_n(x) = Cn^N\rho(nx), \quad \forall x\in\mathbb{R}^N.
\]

It is not difficult to check that this is a sequence of functions in $C_c^{\infty}(\Omega)$ which satisfies \eqref{mol}.

From now, by $\{\rho_n\}$ we wean the above sequence of mollifiers.

Set 
\begin{equation}\label{conv}
\mu_n = \rho_n * \mu ,
\end{equation}
i.e.,
\[
\mu_n(x) = \int_{\Omega} \rho_n(x-y) \,d\mu(y)\quad \forall x\in \mathbb{R}^N.
\]

Here we prove some properties of this sequence of convolutions which will be useful later:

\begin{prn}\label{convthm}
The sequence $\{\mu_n\}$ satisfies:
\begin{enumerate}[i)]
\item $\mu_n \in C_c^{\infty}(\mathbb{R}^N)\quad \forall n$.
\item $\| \mu_n\|_{L^1(\Omega)} \le \| \mu\|_{\mathcal{M}}\quad \forall n$.
\item $\mu_n \overset{*}{\rightharpoonup} \mu\quad \text{weak-$^*$ in } \mathcal{M}(\Omega)$.
\item $\| \mu_n\|_{L^1(\Omega)}  \to \| \mu\|_{\mathcal{M}}$.
\end{enumerate}
\end{prn}

\begin{proof}
\mbox{}

For each $n$, we have
\begin{align*}
D^{\alpha}\mu_n (x) &= D^{\alpha}\int_{\Omega}\rho_n(x-y)\, d\mu(y)\\
&= \int_{\Omega}D^{\alpha}(\rho_n(x-y))\, d\mu(y),
\end{align*}
which implies i).

The result ii) can be obtained by
\begin{align*}
\int_{\Omega}|\mu_n(x)|\,dx&= \int_{\Omega}\left|\int_{\Omega}\rho_n(x-y)\, d\mu(y)\right|\,dx\\
&\le \int_{\Omega}\left(\int_{\Omega}\rho_n(x-y)\, d|\mu|(y)\right)\,dx\\
&= \int_{\Omega}\left(\int_{\Omega}\rho_n(x-y)\,dx\right)\, d|\mu|(y) \quad\text{(by Fubini)}\\
&= \int_{\Omega}\, d|\mu|\\
&= \|\mu\|_{\mathcal{M}}.
\end{align*}

To prove iii), for $f\in C_0(\overline{\Omega})$, we have
\begin{align*}
\int_{\Omega}\mu_n(x) f(x)\, dx &= \int_{\Omega}\left(\int_{\Omega}\rho_n(x-y)f(x)\, d\mu(y)\right)\, dx \\
&= \int_{\Omega}\left(\int_{\Omega}\rho_n(x-y)f(x)\, dx\right)\, d\mu(y)\\
&= \int_{\Omega}\rho_n*f\, d\mu.
\end{align*}

Since
\[
\rho_n*f \to f \quad\text{uniformly on }\overline{\Omega},
\]
we have
\[
\int_{\Omega}\rho_n*f\, d\mu \to \int_{\Omega}f\, d\mu.
\]

In conclusion, we have
\[
\int_{\Omega}\mu_nf \to \int_{\Omega}f\, d\mu,
\]
which is iii).

The result iv) can be obtained by ii) and the following consequence of iii):
\[
\| \mu \|_{\mathcal{M}} \le \liminf \|\mu_n\|_1. \qedhere
\]
\end{proof}

\section{A priori estimates for solutions}
\mbox{}

In translating the result of Brezis-Strauss to measure, we consider the problem:
\begin{equation}\label{mmain}
\left\{
\begin{aligned}
-\Delta u + g(u) & = \mu && \text{in } \Omega , \\
u & = 0 && \text{on } \partial \Omega ,
\end{aligned}
\right.
\end{equation}
with the following three assumptions:
\begin{itemize}
\item $\Omega \subset \mathbb{R}^N$ is a bounded open set with smooth boundary.
\item $g: \mathbb{R} \to \mathbb{R}$ is a continuous, non-decreasing function such that $g(0) = 0$.
\item $\mu \in \mathcal{M}(\Omega)$.
\end{itemize}

We find a solution for the above problem in \emph{weak sense}:
\begin{equation}\label{mweak}
\left\{
\begin{aligned}
& u \in L^1(\Omega ),\quad g(u) \in L^1(\Omega ) ,\\
&-\int_{\Omega} u \Delta \varphi + \int_{\Omega} g(u)\varphi = \int_{\Omega}\varphi\, d\mu \quad \forall \varphi \in C^2_0(\overline{\Omega}) ,
\end{aligned}
\right.
\end{equation}

In case $g \equiv 0$, the linear case, we have the following theorem, which shows the existence, uniqueness and regularity of solutions to our problem. For a proof, see Stampacchia \cite[Th\'eor\`eme 9.1]{S}.
\begin{thm}\label{lithm}
Given $\mu \in \mathcal{M}(\Omega)$, there exists a unique $u\in L^1(\Omega)$ satisfying
\[
-\int_{\Omega}u\Delta \varphi = \int_{\Omega}\varphi \, d\mu \quad \forall \varphi\in C_0^2(\overline{\Omega}).
\]
Moreover, for every $1\le q < \frac{N}{N-1}$, $u\in W^{1,q}_0(\Omega)$ and
\[
\| u\|_{W^{1,q}} \le C \| \mu \|_{\mathcal{M}},
\]
where $C>0$ is a constant depending only on $\Omega$.
\end{thm}

By the same argument as in $L^1$ case, we have the uniqueness of solutions to the problem \eqref{mweak}. Now, we will display some a priori estimates which are useful later.

\begin{thm}\label{apthm}
For every $\mu \in \mathcal{M}(\Omega)$, the problem \eqref{mweak} has at most one solution $u$. Moreover, $u \in W^{1,1}_0(\Omega)$ and satisfies:
\begin{enumerate}[i)]
\item $\| g(u) \|_1 \le \| \mu\|_{\mathcal{M}}$.
\item $\| u \|_1 \le \| u \|_{W^{1,1}} \le C\|\mu\|_{\mathcal{M}}$,\\ where $C>0$ is a constant depending only on $\Omega$.

If $u_1,u_2$ are solutions with measure data $\mu_1,\mu_2$ respectively, then
\item $\|g(u_1) -g(u_2)\|_1 \le \| \mu_1 - \mu_2 \|_{\mathcal{M}}$.
\item $\| u_1 - u_2 \|_1 \le \| u_1 - u_2 \|_{W^{1,1}} \le C\|\mu_1 - \mu_2\|_{\mathcal{M}}$,\\
where $C>0$ is a constant depending only on $\Omega$.
\end{enumerate}
\end{thm}

\begin{proof}
\mbox{}

Let $\{\mu_n\}$ be a sequence of convolutions of $\mu$ as defined in \eqref{conv}.

For each $n$, we have $\mu_n \in C_c^{\infty}(\mathbb{R}^N)$. So,
\[
\mu_n - g(u) \in L^1(\Omega).
\]

Let $u_n$ satisfy
\begin{equation}\label{gg1}
-\int_{\Omega}u_n\Delta\varphi = \int_{\Omega}(\mu_n - g(u))\varphi,\quad \forall \varphi\in C^2_0(\overline{\Omega}).
\end{equation}

By Theorem \ref{lithm}, we have
\begin{equation}\label{gg4}
\| u_n\|_{W^{1,1}} \le C \| \mu_n - g(u)\|_1 \le C(\| \mu \|_{\mathcal{M}} + \| g(u) \|_1),
\end{equation}
which implies $\{ u_n\}$ is a bounded sequence in $W^{1,1}(\Omega)$. By the existence of a compact embedding from $W^{1,1}(\Omega)$ to $L^1(\Omega)$, we can extract a subsequence $\{ u_{n_k}\}$ converging to $v$ in $L^1(\Omega)$. We may suppose more that
\[
u_{n_k} \to v \quad \text{a.e.\ in }\Omega.
\]

By \eqref{gg1}, we have
\[
-\int_{\Omega}u_{n_k}\Delta\varphi + \int_{\Omega} g(u)\varphi = \int_{\Omega}\mu_{n_k}\varphi,\quad \forall \varphi\in C^2_0(\overline{\Omega}).
\]

Taking the limits of both sides gives
\[
-\int_{\Omega}v\Delta \varphi + \int_{\Omega} g(u)\varphi = \int_{\Omega}\varphi\, d \mu \quad\forall \varphi \in C^2_0(\overline{\Omega}),
\]
which implies
\[
\int_{\Omega}(u-v)\Delta \varphi = 0 \quad \forall \varphi \in C^2_0(\overline{\Omega}).
\]

So,
\[
u = v \quad\text{a.e.\ in } \Omega ,
\]
and
\begin{equation}\label{gg2}
u_{n_k} \to u \quad\text{a.e.\ in } \Omega.
\end{equation}

Let $p\in C^2(\mathbb{R})$ such that $p(0)=0$, $p' \ge 0$ and $|p|\le 1$.

By the same argument as in Lemma 1.5, from \eqref{gg1}, we have
\[
\int_{\Omega} (\mu_n - g(u)).p (u_n) \ge 0,
\]
which implies
\[
\int_{\Omega} g(u).p (u_{n_k}) \le \int_{\Omega}\mu_{n_k}.p (u_{n_k}) \le \| \mu_{n_k}\|_1 \le \| \mu \|_{\mathcal{M}}.
\]

Moreover, by the dominated convergence theorem, from \eqref{gg2} we have
\[
\int_{\Omega} g(u).p (u_{n_k}) \to \int_{\Omega} g(u).p (u),
\]
which implies
\begin{equation}\label{gg3}
\int_{\Omega} g(u).p (u) \le \| \mu \|_{\mathcal{M}}.
\end{equation}

Now, let $\{p_n\}$ be a sequence of non-decreasing functions in $C^2(\mathbb{R})$ such that
\[
p_n(t) = \left\{
\begin{aligned}
& 0 \quad &&\text{if } t=0,\\
& 1 \quad &&\text{if } t > \frac{1}{n},\\
& -1 \quad &&\text{if } t < -\frac{1}{n}.
\end{aligned}
\right.
\]

It is clear that
\[
p_n(u)\to \sgn(u)\quad\text{pointwisely in }\Omega,
\]
which implies
\begin{equation}\label{fix1}
g(u).p_n(u)\to g(u).\sgn(u)\quad\text{pointwisely in }\Omega.
\end{equation}

Since $|p_n(u)|\le 1$ for all $n$, we can apply the dominated convergence theorem to \eqref{fix1}. So we get
\[
\int_{\Omega} g(u).p_n (u)\to \int_{\Omega} g(u).\sgn (u).
\]

From \eqref{gg3} and $g(u).\sgn (u) = |g(u)|$, i) follows.

Rewriting \eqref{mweak} gets
\begin{equation}\label{2anh}
-\int_{\Omega}u\Delta\varphi = \int_{\Omega}\varphi\, d\mu  - \int_{\Omega} g(u)\varphi, \quad  \forall \varphi \in C^2_0(\overline{\Omega}).
\end{equation}

Let $v\in L^1(\Omega)$ be the unique solution to the problem:
\begin{equation}\label{1anh}
-\int_{\Omega}v\Delta\varphi = \int_{\Omega}\varphi\, d\mu  - \int_{\Omega} g(u)\varphi, \quad  \forall \varphi \in C^2_0(\overline{\Omega}).
\end{equation}

From \eqref{2anh} and \eqref{1anh}, we have
\[
\int_{\Omega}u\Delta\varphi = \int_{\Omega}v\Delta\varphi, \quad  \forall \varphi \in C^2_0(\overline{\Omega}),
\]

Since
\[
\{\Delta \varphi : \varphi \in C^2_0(\overline{\Omega})\} \subset C_c^{\infty}(\Omega),
\]
we have
\[
u = v\quad \text{a.e.\ in } \Omega,
\]
which means that $u$ is the unique solution in $L^1(\Omega)$ of the problem \eqref{1anh}.

From Theorem \ref{lithm}, we have $u\in W^{1,1}_0(\Omega)$ and
\[
\| u \|_{W^{1,1}} \le C(\| \mu + T_{g(u)} \|_{\mathcal{M}})\le C(\|\mu\|_{\mathcal{M}} + \|g(u)\|_1),
\]
where $T_{g(u)}$ is from \eqref{L12M}.

Hence ii) follows i).

Finally, iii) an iv) can be obtained in the same way as i), with a note that
\[
(g(u_1)-g(u_2)).\sgn(u_1 - u_2) = |g(u_1) - g(u_2)|.\qedhere
\]
\end{proof}

\section{The existence of solutions}
\mbox{}

The answer for the existence of solutions is negative, as in the following example:

\begin{thm}
Assume $N\ge 3$. If $p \ge \frac{N}{N-2}$, then, for any $a\in \Omega$, the problem
\begin{equation}\label{exmain}
\left\{
\begin{aligned}
-\Delta u + |u|^{p-1}u & = \delta_a && \text{in } \Omega , \\
u & = 0 && \text{on } \partial \Omega ,
\end{aligned}
\right.
\end{equation}
has no solution $u\in L^p(\Omega)$.
\end{thm}

\begin{proof}
We prove this problem by contradiction.

Assume that the problem \eqref{exmain} has a solution $u\in L^p(\Omega)$. For convenience, we may suppose $a = 0 \in \Omega$ and $B(0,1) \subset \Omega$.

Let $\varphi \in C_c^{\infty}(\mathbb{R}^N)$ such that $\supp \varphi \subset B(0,1)$. Then let $\{\varphi_n\}\subset C_c^{\infty}(\mathbb{R}^N)$ be the sequence given by
\[
\varphi_n(x) = \varphi(nx), \quad \forall x\in \mathbb{R}^N, \forall n
\]

It is clear that 
\[
\supp \varphi_n \subset B(0, 1/n), \quad \forall n,
\]
and
\begin{equation}\label{lim1}
\lim_{n\to \infty} \varphi_n(x) =\lim_{n\to \infty}\varphi(nx) = 0 ,\quad \forall x \neq 0.
\end{equation}

Using $\varphi_n$ as test function in \eqref{exmain}, we get
\begin{equation}\label{dira1}
-\int_{B(0,1)} u \Delta \varphi_n + \int_{B(0,1)} |u|^{p-1}u\varphi_n = \varphi_n(0) = \varphi (0).
\end{equation}

From \eqref{lim1}, we get
\[
\lim_{n\to \infty} |u|^{p-1}u\varphi_n = 0 \quad\text{a.e.\ in } B(0,1).
\]

Moreover, since
\[
||u|^{p-1}u\varphi_n| \le |u|^p \| \varphi\|_{\infty}.
\]

Applying the dominated convergence theorem, we get
\begin{equation}\label{dira2}
\lim_{n\to \infty} \int_{B(0,1)}|u|^{p-1}u\varphi_n = 0.
\end{equation}

Let $p' = p/(p-1)$, then
\begin{align*}
\left|\int_{B(0,1)} u\Delta \varphi_n\right| &= \left|\int_{B(0,1/n)}u\Delta \varphi_n\right|\\
&= n^2 \left|\int_{B(0,1/n)} u \Delta \varphi (nx)\, dx\right|\\
&\le n^2 \left(\int_{B(0,1/n)} |u|^p\right)^{1/p}\left(\int_{B(0,1/n)} |\Delta\varphi (nx)|^{p'}\,dx\right)^{1/p'}\\
&= n^2 \left(\int_{B(0,1/n)} |u|^p\right)^{1/p}\left(n^{-N}\int_{B(0,1/n)} |\Delta\varphi (nx)|^{p'}\,dx\right)^{1/p'}\\
&= n^{2-N/p'} \left(\int_{B(0,1/n)} |u|^p\right)^{1/p}\left(\int_{B(0,1)} |\Delta\varphi|^{p'}\right)^{1/p'}.
\end{align*}

Since $p \ge \frac{N}{N-2}$, we have $p' \le N/2$. Hence
\[
n^{2-N/p'} \le 1,\quad \forall n.
\]

So,
\[
\left|\int_{B(0,1)} u\Delta \varphi_n\right| \le \left(\int_{B(0,1/n)} |u|^p\right)^{1/p}\left(\int_{B(0,1)} |\Delta\varphi|^{p'}\right) ^{1/p'},\quad\forall n.
\]

By the dominated convergence theorem, we have
\[
\int_{B(0,1/n)} |u|^p = \int_{\Omega}\chi_{B(0,1/n)}|u|^p \to 0,
\]
which implies
\begin{equation}\label{dira3}
\int_{B(0,1)} u\Delta \varphi_n \to 0.
\end{equation}

From \eqref{dira1}, \eqref{dira2}, and \eqref{dira3}, we conclude that
\[
\varphi(0) = 0.
\]

To reach a contradiction, pick $\varphi$ such that $\varphi(0) \neq 0$.
\end{proof} 

Despite of this counterexample, when $g$ is good enough, the answer for the existence of solutions is positive. The following theorem demonstrates a large class of $g$ satisfying the existence. In some sense, it also shows that the value $\frac{N}{N-2}$ in the above example is critical.

\begin{thm}
Assume $N \ge 2$ and
\[
|g(t)| \le C(|t|^p + 1), \quad\forall t\in \mathbb{R},
\]
for some $p \in [1, \frac{N}{N-2})$. Then, for every $\mu\in \mathcal{M}(\Omega)$, problem \eqref{mweak} has a unique solution.
\end{thm}

\begin{proof}
\mbox{}

Let $\{\mu_n\}$ be a sequence of convolutions of $\mu$ as in \eqref{conv}.

For each $n$, we have $\mu_n \in C_c^{\infty}(\mathbb{R}^N)$, so there is $u_n\in L^1(\Omega)$ satisfies
\begin{equation}\label{eee3}
-\int_{\Omega}u_n\Delta\varphi + \int_{\Omega}g(u_n)\varphi = \int_{\Omega}\mu_n \varphi\quad \forall \varphi\in C^2_0(\overline{\Omega}),
\end{equation}
or equivalently,
\[
-\int_{\Omega}u_n\Delta\varphi = \int_{\Omega}(\mu_n - g(u_n)) \varphi\quad \forall \varphi\in C^2_0(\overline{\Omega}).
\]

For $p < \frac{N}{N-2}$, we have 
\[
\frac{1}{1/p + 1/N} <  \frac{N}{N-1}.
\]

Let $q$ be such that 
\[
\frac{1}{1/p + 1/N} < q < \frac{N}{N-1}.
\]

By Theorem \ref{lithm}, we have
\[
\| u_n \|_{W^{1,q}} \le C_1\| \mu_n - g(u_n) \|_1,
\]
and also by Theorem \ref{apthm},
\[
\| u_n \|_{W^{1,q}} \le 2C_1\| \mu_n \|_1 \le 2C_1\| \mu \|_{\mathcal{M}}.
\]

We conclude that the sequence $\{u_n\}$ is bounded in $W^{1,q}(\Omega)$.

Since $q < \frac{N}{N-1}\le N$, there is a compact embedding from $W^{1,q}(\Omega)$ to $L^p(\Omega)$. So there exists a subsequence $\{u_{n_k}\}$ converging in $L^p(\Omega)$ to $u$. 

We may suppose furthermore that
\begin{equation}\label{e111}
u_{n_k} \to u \quad\text{a.e.\ in } \Omega,
\end{equation}
and there exists some function $h\in L^p(\Omega)$ such that
\[
|u_{n_k}| < h \quad \text{a.e.\ in }\Omega.
\]

By \eqref{e111}, we have
\[
g(u_{n_k}) \to g(u) \quad \text{a.e.\ in } \Omega.
\]

Moreover,
\[
|g(u_{n_k})| \le C(|u_{n_k}|^p + 1) \le C(h^p + 1).
\]

Since $C(h^p + 1) \in L^1(\Omega)$, applying the dominated convergence theorem gives
\[
g(u_{n_k}) \to g(u) \quad \text{in } L^1(\Omega).
\]

Taking the limits of both sides in \eqref{eee3}, we get the conclusion.
\end{proof}

\chapter{Some preliminaries}

\section{Capacity}
\mbox{}

Hereafter, the notion \emph{capacity} will stand for the \emph{Newtonian} capacity (or $H^1$-capacity), which is a function of Borel sets with value in $[0,+\infty]$.

Given a compact subset $K$ of $\Omega$, we denote the capacity  of $K$ with respect to $\Omega$ by
\begin{equation}\label{capdef}
\capa(K) \equiv \capa_{H^1}(K) = \inf \left\{ \int_{\Omega} |\nabla \varphi |^2: \varphi\in C_c^{\infty}(\Omega), K \subset [\varphi\ge 1]^o \right\},
\end{equation}
with the convention that $\inf \emptyset = +\infty$.

By $K \subset [\varphi\ge 1]^o$ we mean that $\varphi \ge 1$ on some neighborhood of $K$.

If $U$ is an open subset of $\Omega$, we denote
\begin{equation}\label{312}
\capa(U) = \sup \{\capa(K): K\text{ compact and } K\subset U \}.
\end{equation}

Finally, for any Borel subset $E$ of $\Omega$, we denote
\begin{equation}\label{313}
\capa(E) = \inf \{\capa(U): U\text{ open and } E \subset U\}.
\end{equation}

Let $K_1, K_2$ be two compact subsets of $\Omega$. We have
\begin{equation}\label{314}
\capa(K_1) \le \capa(K_2) \quad\text{if }K_1\subset K_2.
\end{equation}

By \eqref{314}, we can show that the definition \eqref{313} agrees with \eqref{capdef} and \eqref{312} when $E$ is a compact subset of $\Omega$ and an open subset of $\Omega$ respectively.

The inequality \eqref{314} also implies that
\begin{equation}\label{cap1}
\capa(A) \le \capa(B) \quad \text{if }A\subset B.
\end{equation}

It is clear that $\capa (\emptyset) = 0$.

Moreover, for $K$ compact and for any $\varphi \in C_c^{\infty}(\Omega)$ such that $\varphi \ge 1$ on a neighborhood of $K$, by Poincar\'e's Inequality, we have 
\[
|K| \le \int_{\Omega}|\varphi|^2 \le C \int_{\Omega} |\nabla \varphi |^2,
\]
which implies that
\[
|K| \le C \capa(K),
\]
where $C > 0$ is a constant just depending on $\Omega$. 

Using the above estimate, we can show that if $\capa(E) = 0$, then $|E| = 0$ for every $E$ be a Borel subset of $\Omega$.

This capacity also has subadditivity property, which is:

\begin{prn}\label{cap2thm}
Let $\{A_n\}$ be a sequence of subsets of $\Omega$. We have
\begin{equation}\label{cap2}
\capa\left(\bigcup\limits_{n=1}^{\infty}A_n\right) \le \sum\limits_{n=1}^{\infty}\capa(A_n).
\end{equation}
\end{prn}\label{311prn}

\begin{proof}
By \eqref{313}, we can simplify our proof to validate \eqref{cap2} in case $A_n$ is an open set for all $n$.

Let $K$ be a compact subset of $\bigcup\limits_{n=1}^{\infty} A_n$. Then there exist finite sets $A_{n_1},A_{n_2},\ldots ,A_{n_k}$ in sequence $\{A_n\}$ such that
\[
K\subset \bigcup\limits_{i=1}^{k} A_{n_i}.
\]

If we can show that
\begin{equation}\label{cap4}
\capa(K) \le \sum\limits_{i=1}^{k}\capa(A_{n_i})
\end{equation}
then by
\[
\sum\limits_{i=1}^{k}\capa(A_{n_i}) \le \sum\limits_{n=1}^{\infty}\capa(A_n),
\]
and \eqref{312}, we get the conclusion.

Instead of proving \eqref{cap4}, it is enough to show that if $K$ is a compact subset of $A \cup B$ with $A, B$ are two open sets, then $\capa(K) \le \capa(A) + \capa(B)$.

For each $x \in K \cap A$, we can pick a neighborhood $U_x$ of $x$ such that $\overline{U}_x \subset A$. Similarly, for each $y \in K \cap B$, we can pick an neighborhood $V_y$ of $y$ such that $\overline{V}_y \subset A$. Now we have
\[
K \subset \left(\bigcup\limits_{x\in K\cap A}U_x \right)\cup \left(\bigcup\limits_{y\in K\cap B}V_y\right).
\]
By the compactness of $K$, there exists finite points $x_1, x_2,\ldots, x_p\in A$ and $y_1,y_2,\ldots,y_q\in B$ such that
\[
K \subset  \left(\bigcup\limits_{i=1}^{p} U_{x_i} \right)\cup  \left( \bigcup\limits_{j=1}^{q} V_{y_j}\right) .
\]
Let $C = \bigcup\limits_{i=1}^{p} \overline{U}_{x_i}$ and $D = \bigcup\limits_{j=1}^{q} \overline{U}_{y_j}$. We have $K \subset C \cup D$, where $C,D$ are two compact subsets of $A$ and $B$ respectively. If we can show that
\begin{equation}\label{cap5}
\capa (C \cup D) \le \capa(C) + \capa(D),
\end{equation}
then we will have
\[
\capa(K) \le \capa(C\cup D) \le \capa(C) + \capa(D) \le \capa(A) + \capa(B),
\]
which gives us the conclusion.

Now, the remaining thing is to prove \eqref{cap5}.

Let $\varphi_1,\varphi_2\in C_c^{\infty}(\Omega)$ be such that $\varphi_1 \ge 1$ on a neighborhood of $C$ and $\varphi_2\ge 1$ on a neighborhood of $D$; and let $\psi = \max \{\varphi_1, \varphi_2\}$.

It is clear that $\psi \ge 1$ on a neighborhood of $C \cup D$. For $n$ large enough, the function $\psi_n  = \rho_n * \psi$ belongs to $C_c^{\infty}(\Omega)$ and $\psi_n \ge 1$ on a neighborhood of $C \cup D$. So,
\begin{equation}\label{cap10}
\capa(C\cup D) \le \int_{\Omega}|\nabla \psi_n|^2.
\end{equation}

For each $1 \le i \le N$, we have
\[
\| \frac{\partial \psi_n}{\partial x_i} \|_2=\| \rho_n * \frac{\partial \psi}{\partial x_i} \|_2 \le \| \rho_n \|_1 \|\frac{\partial \psi}{\partial x_i}\|_2 =  \|\frac{\partial \psi}{\partial x_i}\|_2  ,
\]
which implies
\begin{equation}\label{cap11}
\int_{\Omega}|\nabla \psi_n|^2 \le \int_{\Omega}|\nabla \psi|^2 .
\end{equation}

Now, since
\[
\nabla \psi  = \left\{
\begin{aligned}
&\nabla \varphi_1 &&\text{a.e.\ on }[\varphi_1 \ge \varphi_2],\\
&\nabla \varphi_2 &&\text{a.e.\ on }[\varphi_1 \le \varphi_2],
\end{aligned}
\right.
\]
we have
\begin{equation}\label{cap12}
|\nabla \psi |^2 \le |\nabla \varphi_1|^2 + |\nabla \varphi_2|^2\quad\text{a.e.\ on }\Omega.
\end{equation}

Combining \eqref{cap10}, \eqref{cap11}, and \eqref{cap12}, we get
\[
\capa(C\cup D) \le \int_{\Omega}|\nabla \varphi_1|^2 + \int_{\Omega}|\nabla \varphi_2|^2,
\]
which gives us the conclusion because $\varphi_1, \varphi_2$ are chosen arbitrarily.
\end{proof}

The following proposition gives us a finer definition of capacity comparing to \eqref{capdef}.

\begin{prn}\label{31prn}
For every compact set $K\subset \Omega$, we have
\[
\capa (K)= \inf \left\{ \int_{\Omega} |\nabla \varphi |^2: \varphi\in C_c^{\infty}(\Omega), 0\le \varphi\le 1, K \subset [\varphi = 1]^o \right\}.
\]
\end{prn}

By $K\subset [\varphi = 1]^o$, we mean that $\varphi = 1$ on some neighborhood of $K$.

\begin{proof}
Denote the right hand side by $T$.

Clearly it is enough to show that
\[
\capa (K) \ge \inf \left\{ \int_{\Omega} |\nabla \varphi |^2: \varphi\in C_c^{\infty}(\Omega), 0\le \varphi\le 1, K \subset [\varphi = 1]^o \right\}.
\]

Indeed, let $\varphi \in C_c^{\infty}(\Omega)$ be such that $\varphi \ge 1$ in some neighborhood of $K$ and let $\psi = \min\{\varphi^+,1\}$.

It is clear that $0\le \psi\le 1$ and $\psi = 1$ on a neighborhood of $K$. For $n$ large enough, the function $\psi_n = \rho_n * \psi$ belongs to $C_c^{\infty}(\Omega)$, $0\le \psi_n \le 1$, and $\psi_n = 1$ on a neighborhood of $K$. So,
\begin{equation}\label{cap121}
T \le \int_{\Omega}|\nabla \psi_n|^2.
\end{equation}

For each $1 \le i \le N$, we have
\[
\| \frac{\partial \psi_n}{\partial x_i} \|_2 =\| \rho_n * \frac{\partial \psi}{\partial x_i} \|_2 \le \| \rho_n \|_1 \|\frac{\partial \psi}{\partial x_i}\|_2 =  \|\frac{\partial \psi}{\partial x_i}\|_2  ,
\]
which implies
\begin{equation}\label{cap122}
\int_{\Omega}|\nabla \psi_n|^2 \le \int_{\Omega}|\nabla \psi|^2 
\end{equation}

Now, from
\[
\nabla \psi  = \left\{
\begin{aligned}
&0 &&\text{a.e.\ on }[\varphi \le 0]\cup[\varphi \ge 1],\\
&\nabla \varphi &&\text{a.e.\ on }[0 \le \varphi \le 1],
\end{aligned}
\right.
\]
we have
\begin{equation}\label{cap123}
|\nabla \psi|^2 \le |\nabla \varphi |^2 \quad\text{a.e.\ on }\Omega.
\end{equation}

Combining \eqref{cap121}, \eqref{cap122} and \eqref{cap123}, we get
\[
T \le \int_{\Omega} |\nabla \psi |^2,
\]
which gives us the conclusion because $\varphi$ is chosen arbitrarily.
\end{proof}

Another definition of capacity for a compact subset $K$ of $\Omega$ is
\[
\capa_{\Delta,1}(K) = \inf \left\{\int_{\Omega} |\Delta \varphi |: \varphi\in C_c^{\infty}(\Omega),K \subset [\varphi \ge 1]^o \right\},
\]
or (similar to Proposition \ref{31prn})
\[
\capa_{\Delta,1}(K) = \inf \left\{\int_{\Omega} |\Delta \varphi |: \varphi\in C_c^{\infty}(\Omega), 0\le\varphi\le 1, K \subset [\varphi = 1]^o\right\}.
\]

This definition and our definition are equivalent, as shown in the following theorem:

\begin{thm}
For every compact set $K \subset \Omega$, we have
\[
\capa_{\Delta, 1}(K) = 2 \capa_{H^1}(K).
\]
\end{thm}

First we need the following lemma whose proof requires some knowledge in Potential Theory (see the main article of this report \cite[Lemma E.1]{BMP}).

\begin{lmm}\label{caplmm}
Let $K \subset \Omega$ be a compact set. Given $\epsilon > 0$, there exists $\psi \in C_c^{\infty}(\Omega)$ such that $0\le \psi \le 1$ in $\Omega$, $\psi = 1$ in some neighborhood of $K$, and
\[
\int_{\Omega}|\Delta \varphi| \le 2 \capa_{H^1}(K) + \epsilon.
\]
\end{lmm}

\begin{proof}[Proof of Theorem 3.1]
Clearly by Lemma \ref{caplmm}, it is enough to prove that
\[
\capa_{H^1}(K) \le \frac{1}{2}\capa_{\Delta,1}(K).
\]

Indeed, let $\varphi \in C_c^{\infty}(\Omega)$ be such that $0\le \varphi \le 1$ in $\Omega$ and $\varphi = 1$ in some neighborhood of $K$. We have
\[
\capa_{H^1}(K) \le \int_{\Omega}|\nabla \varphi|^2 = -\int_{\Omega}\varphi \Delta \varphi.
\]

Because $\varphi$ has compact support in $\Omega$ and $0\le \varphi \le 1$, we have
\[
\int_{\Omega}\Delta \varphi = 0\quad\text{and} \quad |\varphi - \frac{1}{2}| \le \frac{1}{2},
\]
which imply
\[
\capa_{H^1}(K) \le -\int_{\Omega}(\varphi - \frac{1}{2})\Delta \varphi \le \frac{1}{2}\int_{\Omega}|\Delta \varphi|.
\]

We get the conclusion because $\varphi$ is chosen arbitrarily.
\end{proof}

\section{Decomposition of measures}
\mbox{}

We start this section with the following result:

\begin{thm}\label{33thm}
Let $\mu$ be a bounded Borel measure in $\mathbb{R}^N$ and let $\mathcal{Z}$ be a collection of Borel sets such that
\begin{enumerate}[i)]
\item $\mathcal{Z}$ is closed with respect to finite or countable unions.
\item $(A\in \mathcal{Z} \text{ and } A'\subset A \text{ Borel}) \quad \Rightarrow \quad A' \in \mathcal{Z}$.
\end{enumerate}
Then $\mu$ can be represented in the form
\[
\mu = \mu_1 + \mu_2,
\]
where $\mu_1$ and $\mu_2$ are bounded Borel measures such that
\[
|\mu_1|(A) = 0, \quad \forall A \in \mathcal{Z},
\]
and
\[
|\mu_2|(A_0^c) = 0 \quad \text{for some }A_0 \in \mathcal{Z}.
\]
\end{thm}

\begin{proof}
\mbox{}

First, we assume that $\mu$ is nonnegative.

Denote
\[
X_{\mu} = \sup \{\mu(A): A\in \mathcal{Z}\}.
\]

Because $\mu$ is a finite measure, we have $X_{\mu}$ finite. Let $\{A_n\}$ be a sequence of sets in $\mathcal{Z}$ such that
\[
\mu(A_n) \uparrow X_{\mu}.
\]

Let $A_0 = \bigcup\limits_{n=1}^{\infty} A_n$. From i) and ii), we have
\begin{equation}\label{331}
A_0 \in \mathcal{Z} \quad\text{and}\quad\mu(A_0) = X_{\mu}.
\end{equation}

For every Borel set $E$, we put
\[
\mu_1(E) = \mu(E\setminus A_0),\quad \mu_2(E) = \mu(E\cap A_0).
\]

It is clear that $\mu_2(A_0^c) = 0$. Moreover, for every $E\in \mathcal{Z}$, we have
\[
\mu(A_0\cup E) = X_{\mu}\quad\text{(by } A_0 \cup E \in \mathcal{Z}\text{, (3.4) and ii))},
\]
and
\[
\mu(A_0 \cup E) = \mu(A_0 \cup (E \setminus A_0)) = \mu(A_0) + \mu(E\setminus A_0) = X_{\mu} + \mu_1(E),
\]
which imply that $\mu_1(E) = 0$.

Now, when $\mu$ is a signed measure, applying the above result to $\mu^+$ and $\mu^-$, we get
\[
\mu = \mu^+ - \mu^- = (\mu^+)_1 + (\mu^+)_2 - (\mu^-)_1 - (\mu^-)_2,
\]
where
\[
(\mu^+)_1(A) = (\mu^-)_1(A) = 0, \quad \forall A \in \mathcal{Z},
\]
and 
\[
(\mu^+)_2(B^c) = (\mu^-)_2(C^c) = 0 \quad \text{for some }B,C \in \mathcal{Z}.
\]

Put $\mu_1 = (\mu^+)_1 - (\mu^-)_1$, $\mu_2 = (\mu^+)_2 - (\mu^-)_2$. We have $B \cup C \in \mathcal{Z}$,
\[
|\mu_1|(A) = (\mu^+)_1(A) + (\mu^-)_1(A) = 0, \quad \forall A\in\mathcal{Z},
\]
and
\begin{align*}
|\mu_2|((B\cup C)^c) &= (\mu^+)_2((B\cup C)^c) + (\mu^-)_2((B\cup C)^c)\\
&\le (\mu^+)_2(B^c) + (\mu^-)_2(C^c)\\
&= 0.
\end{align*}

The last one implies that
\[
|\mu_2|((B\cup C)^c) = 0.
\]

Finally, if $\mu = \mu_1 + \mu_2$ and $\mu = \mu_1' + \mu_2'$ are two decompositions such that
\[
|\mu_1|(A) = |\mu_1'|(A) = 0, \quad \forall A\in \mathcal{Z},
\]
and
\[
|\mu_2|(B^c) = |\mu_2'|(C^c) = 0 \quad \text{for some }B, C\in \mathcal{Z},
\]
then for every Borel set $E$,
\begin{align*}
\mu_2(E) &= \mu_2(E\cap (B\cup C)) + \mu_2(E\cap (B\cup C)^c)\\
&= \mu_2 (E\cap (B\cup C))\quad \text{(by }E\cap (B\cup C)^c \subset B^c\text{)}\\
&= \mu(E\cap (B\cup C)) - \mu_1(E\cap (B\cup C))\\
&= \mu(E\cap (B\cup C))\\
&= \mu_2'(E),
\end{align*}
which implies the uniqueness.
\end{proof}

The main content of this section is the following decomposition of a measure into ``diffuse'' and ``concentrated'' parts. We can show that this decomposition is a consequence of \eqref{cap1}, \eqref{cap2} and Theorem \ref{33thm} by putting
\[
\mathcal{Z} = \{E \subset \Omega:  E\text{ Borel and } \capa(E) = 0\}.
\]


\begin{crr}\label{331crr}
Any measure $\mu \in \mathcal{M}(\Omega)$ can be uniquely decomposed as a sum of two measures:
\[
\mu = \mu_d + \mu_c,
\]
where $|\mu_d|(E)=0$ for every Borel set $E \subset \Omega$ such that $\capa (E) = 0$ and $|\mu_c|(\Omega \setminus F)$ for some Borel set $F \subset \Omega$ such that $\capa (F) = 0$.
\end{crr}

\begin{rmk}
By the uniqueness, we can show that
\[
(\mu \pm \nu)_d = \mu_d \pm \nu_d \quad \text{and} \quad (\mu \pm \nu)_c = \mu_c \pm \nu_c.
\]
It is also clear that $\mu_d$ and $\mu_c$ are mutual singular with respect to each other. So we have
\[
(\mu^+)_d = (\mu_d)^+, \quad (\mu^-)_d = (\mu_d)^-,\quad (\mu^+)_c = (\mu_c)^+, \quad\text{and} \quad (\mu^-)_c = (\mu_c)^-.
\]
For convenience, we denote these quantities by $\mu^+_d$, $\mu^-_d$, $\mu^+_d$, and $\mu^+_c$ respectively. Moreover, since $\nu - \mu$ is a positive measure if $\mu \le \nu$, we have
\[
\mu \le \nu \quad\Rightarrow\quad (\mu_d \le \nu_d \text{ and } \mu_c \le \nu_c).
\]
We end this remark with
\[
\sup\{\mu_d, \mu_c\} = \mu_c + (\mu_d - \mu_c)^+ = \mu_c + \mu_d^+ + \mu_c^- = \mu_c^+ + \mu_d^+ = \mu^+.
\]
\end{rmk}

\section{Diffuse measures}
\mbox{}

In this section, we concentrate on the ``diffuse'' part of the decomposition in Corollary \ref{331crr}. A measure $\mu \in \mathcal{M}(\Omega)$ is called \emph{diffuse} if $|\mu|(A) = 0$ for every Borel set $A \subset \Omega$ such that $\capa(A) =0$.

Hereafter, we denote by $\mathcal{M}_d(\Omega)$ the set of diffuse measures.

Diffuse measures have many interesting properties when study semilinear elliptic partial differential equations involving measure. There is some ways to define it. In the work of Boccardo-Gallou\"et-Orsina \cite{BGO}, they show that a measure $\mu$ is diffuse if and only if $\mu\in L^1 + H^{-1}$, which is there exist $f_0 \in L^1(\Omega)$ and $v_0 \in H^1_0(\Omega)$ such that
\[
\int_{\Omega}\varphi \,d\mu = \int_{\Omega} f_0\varphi - \int_{\Omega}\nabla v_0 .\nabla \varphi, \quad\forall \varphi \in C_0(\overline{\Omega})\cap H^1_0(\Omega).
\]

One interesting consequence of the above decomposition is the existence of solutions to our problem when data is a diffuse measure. Indeed, we will show that our problem has a unique solution for every data is a ``distribution'' in $L^1 + H^{-1}$. At first, we need the following lemma:

\begin{lmm}\label{33lmm}
For every $T\in H^{-1}(\Omega)$, the equation
\[
\int_{\Omega}\nabla u.\nabla \varphi + \int_{\Omega} g(u)\varphi = \langle T , \varphi \rangle, \quad \forall \varphi \in H^1_0(\Omega) \cap L^{\infty}(\Omega)
\]
has a solution $u\in H^{1}_0(\Omega)$ such that $g(u)\in L^1(\Omega)$.
\end{lmm}

\begin{proof}
The proof of this lemma is the same as the proof of Lemma \ref{L2glmm} and Lemma \ref{L2lmm}, with a note that
\[
|\langle T,\varphi \rangle| \leq \| T \|_{H^{-1}}\|\nabla\varphi\|_2,\quad\forall \varphi \in C^2_0(\overline{\Omega}).\qedhere
\]
\end{proof}

\begin{thm}
For every $f \in L^1(\Omega)$ and every $T\in H^{-1}(\Omega)$, the equation
\[
\int_{\Omega}\nabla u.\nabla \varphi + \int_{\Omega} g(u)\varphi = \int_{\Omega} f\varphi + \langle T , \varphi \rangle ,\quad \forall \varphi \in C^2_0(\overline{\Omega})
\]
has a unique solution $u\in L^1(\Omega)$ such that $g(u)\in L^1(\Omega)$.
\end{thm}

\begin{proof}
\mbox{}

Let $\{f_n\}$ be a sequence in $C^{\infty}_c(\Omega)$ which converges to $f$ in $L^1(\Omega)$. So, for each $n$, $f_n + T$ is in $H^{-1}(\Omega)$.

By Lemma \ref{33lmm}, for each $n$, there exists $u_n\in H^1_0(\Omega)$ with $g(u_n)\in L^1(\Omega)$ such that
\[
-\int_{\Omega} u_n\Delta \varphi +\int_{\Omega} g(u_n)\varphi = \int_{\Omega} f_n \varphi + \langle T , \varphi \rangle, \quad\forall \varphi\in C^2_0(\overline{\Omega}).
\]

So, for each $m,n$, we have
\[
-\int_{\Omega} (u_m-u_n)\Delta \varphi = \int_{\Omega} (f_m-f_n-g(u_m)+g(u_n)) \varphi, \quad\forall \varphi\in C^2_0(\overline{\Omega}).
\]

The remaining proof is the same as in the proof of Theorem \ref{L1thm}.
\end{proof}

Here is a direct consequence of the above theorem.

\begin{crr}\label{33crr}
Let $\mu$ be a diffuse measure. Then the problem
\[
-\int_{\Omega} u \Delta \varphi + \int_{\Omega} g(u)\varphi = \int_{\Omega}\varphi\, d\mu, \quad \forall \varphi \in C^2_0(\overline{\Omega})
\]
has a unique solution $u\in L^1(\Omega)$ with $g(u)\in L^1(\Omega)$.
\end{crr}

\section{Kato's Inequality}
\mbox{}

For every Borel subset $E$ of $\Omega$, we denote by $\mu\lfloor_E$ the measure defined by $\mu\lfloor_E(A) = \mu(A \cap E)$ for every Borel set $A \subset \Omega$.

In this section, we display some versions of Kato's Inequality. The first version is the following theorem, due to Brezis-Ponce \cite[Theorem 1.1]{BP}.

\begin{thm}\label{kato1thm}
Let $u\in L^1(\Omega)$ be such that $\Delta u \in \mathcal{M}(\Omega)$. Then, for every open set $\omega \subset\subset \Omega$, $\Delta u^+$ is a measure on $\omega$ with $(\Delta u^+)\lfloor_{\omega} \in \mathcal{M}(\omega)$ and the following hold:
\begin{equation}\label{kato1}
(\Delta u^+)_d \ge \chi_{[u \ge 0]}(\Delta u)_d \quad\text{in }\omega,
\end{equation}
\begin{equation}\label{kato2}
(-\Delta u^+)_c = (-\Delta u)_c^+ \quad\text{in }\omega.
\end{equation}
\end{thm}

Here by $\Delta u \in \mathcal{M}(\Omega)$ we mean that there exists $\mu \in \mathcal{M}(\Omega)$ such that $\Delta u = \mu$ in $[C^2_0(\overline{\Omega})]^*$; more precisely,
\[
\int_{\Omega}u\Delta \varphi = \int_{\Omega}\varphi\,d\mu \quad\forall \varphi \in C^2_0(\overline{\Omega}).
\]

\begin{rmk}
The results \eqref{kato1} and \eqref{kato2} should be understood in the way that for every Borel subset $E$ of $\omega$,
\[
(\Delta u^+)_d (E) \ge (\Delta u)_d(E \cap [u\ge 0]),
\]
\[
(-\Delta u^+)_c (E) = (-\Delta u)_c^+ (E).
\]
Equivalently, by Proposition \ref{21prn}, \eqref{kato1} and \eqref{kato2} can be understood in the sense of distributions, which is in $\mathcal{D}'(\omega)$. Since $\omega$ is chosen arbitrarily, we also have
\[
(\Delta u^+)_d \ge \chi_{[u \ge 0]}(\Delta u)_d, \quad\text{in }\mathcal{D}'(\Omega),
\]
\[
(-\Delta u^+)_c = (-\Delta u)_c^+, \quad\text{in }\mathcal{D}'(\Omega).
\]
Now it may occur that when $u = v$ a.e.\ in $\Omega$, $(\Delta u)_d(E \cap [u\ge 0])$ and $(\Delta v)_d(E \cap [v\ge 0])$ are different. To get over this complication, we systematically replace $u$ by its quasi-continuous representative. Recall that $u$ is quasi-continuous if and only if, given any $\epsilon > 0$, one can find an open set $\omega_{\epsilon} \subset \Omega$ such that $\capa(\omega_{\epsilon}) < \epsilon$ and $u|_{\Omega \setminus \omega_{\epsilon}}$ is continuous. The quasi-continuous representatives always exists when $u\in L^1(\Omega)$ such that $\Delta u \in \mathcal{M}(\Omega)$, see \cite[Lemma 1]{BP2}. Moreover, we also have the uniqueness of quasi-continuous representatives with respect to capacity, which means that if $u_1,u_2$ are two representatives, then $u_1 = u_2$ quasi-everywhere in $\Omega$ (= outside a set of zero capacity). Indeed, let $u_1,u_2$ be two quasi-continuous functions such that $u_1 = u_2$ a.e.\ in $\Omega$. For each $\epsilon > 0$, there exists an open set $\omega_{\epsilon} \subset \Omega$ with $\capa(\omega_{\epsilon}) < \epsilon$ such that $u_1|_{\Omega \setminus \omega_{\epsilon}}$ and $u_2|_{\Omega \setminus \omega_{\epsilon}}$ are continuous. Since $u_1 = u_2$ a.e.\ in $\Omega$, we obtain
\[
u_1 = u_2 \quad\text{in }\Omega \setminus \overline{\omega_{\epsilon}},
\]
which also implies
\[
u_1 = u_2 \quad\text{in }\Omega \setminus \omega_{\epsilon}
\]
due to the continuity of $u_1|_{\Omega \setminus \omega_{\epsilon}}$ and $u_2|_{\Omega \setminus \omega_{\epsilon}}$. Set
\[
\omega = \bigcap\limits_{n = 1}^{\infty} \omega_{1/n}.
\]
Thus, $\capa(\omega) = 0$ and $u = v$ on $\Omega \setminus \omega$. The uniqueness follows. Now, since $(\Delta u)_d \in \mathcal{M}_d(\Omega)$, we have $(\Delta u)_d(E \cap [u\ge 0])$ well-defined. We end this remark with a note that when $u$ is quasi-continuous, the set $[u = \pm \infty]$ has zero capacity. Its proof can be obtained by a similar argument as in proving the uniqueness.
\end{rmk}

Here is another version of Kato's Inequality. A proof of its which adapts from Theorem \ref{kato1thm} can be found in the main article of this report, see \cite[Proposition B.5]{BMP}.

\begin{thm}\label{kato2thm}
Let $u\in L^1(\Omega)$ and $f\in L^1(\Omega)$ be such that
\[
-\int_{\Omega}u\Delta \varphi \le \int_{\Omega} f \varphi, \quad \forall \varphi \in C^2_0(\overline{\Omega}),\varphi\ge 0.
\]
Then
\[
-\int_{\Omega}u^+\Delta \varphi \le \int_{[u\ge 0]} f \varphi, \quad \forall \varphi \in C^2_0(\overline{\Omega}),\varphi\ge 0.
\]
\end{thm}

We conclude this section with two consequences of Theorem \ref{kato1thm}. The first one is clear due to \eqref{kato2}.

\begin{crr}\label{katocrr1}
Let $u \in L^1(\Omega)$ be such that $\Delta u \in \mathcal{M}(\Omega)$. If $u\ge 0$ a.e.\ in $\Omega$, then
\[
(-\Delta u)_c \ge 0 \quad\text{in }\Omega.
\]
\end{crr}

\begin{crr}\label{katocrr2}
Let $u \in L^1(\Omega)$ be such that $\Delta u \in \mathcal{M}(\Omega)$. Then,
\[
\Delta T_k(u) \le \chi_{[u\le k]}(\Delta u)_d + (\Delta u)^+_c \quad \text{in }\mathcal{D}'(\Omega).
\]
\end{crr}

Here for every $s\in \mathbb{R}$,
\[
T_k(s) = \left\{
\begin{aligned}
&s & \text{if }s\leq k,\\
&k & \text{if }s > k.
\end{aligned}
\right.
\]

\begin{proof}
\mbox{}

It is clear that $T_k(u) = k - (k - u)^+$.

Applying Theorem \ref{kato1thm} for $k-u$ instead of $u$, we have
\[
(\Delta (k-u)^+)_d \ge \chi_{[k-u\ge 0]} (\Delta (k-u))_d = -\chi_{[u\le k]}( (\Delta u)_d \quad\text{in } \mathcal{D}'(\Omega),
\]
\[
(-\Delta (k-u)^+)_c = (-\Delta(k-u))_c^+ = (\Delta u)_c^+ \quad \text{in }\mathcal{D}'(\Omega),
\]
which imply that
\begin{align*}
\Delta T_k(u) &= (\Delta T_k(u))_d + (\Delta T_k(u))_c\\
&=(-\Delta (k-u)^+)_d + (-\Delta (k-u)^+)_c \\
&\le \chi_{[u\le k]}(\Delta u)_d + (\Delta u)_c^+\quad \text{in }\mathcal{D}'(\Omega).\qedhere
\end{align*}
\end{proof}

\chapter{Reduced measures}

\section{\texorpdfstring{Constructions of $u^*$ and $\mu^*$}{Constructions of u* and mu*}}
\mbox{}

Returning to our problem:
\begin{equation}\label{4main}
\left\{
\begin{aligned}
-\Delta u + g(u) & = \mu && \text{in } \Omega , \\
u & = 0 && \text{on } \partial \Omega.
\end{aligned}
\right.
\end{equation}

We call $u$ a \emph{solution} of \eqref{4main} if
\begin{equation}\label{4weak}
\left\{
\begin{aligned}
& u \in L^1(\Omega ),\quad g(u) \in L^1(\Omega ) ,\\
&-\int_{\Omega} u \Delta \varphi + \int_{\Omega} g(u)\varphi = \int_{\Omega}\varphi\, d\mu, \quad \forall \varphi \in C^2_0(\overline{\Omega}) .
\end{aligned}
\right.
\end{equation}

For convenience, we write
\[
-\Delta u +g(u) = \mu \quad\text{in }[C^2_0(\overline{\Omega})]^*,
\]
if $u$ satisfies \ref{4weak}.

We call $u$ a \emph{subsolution} of \eqref{4main} if
\[
\left\{
\begin{aligned}
& u \in L^1(\Omega ),\quad g(u) \in L^1(\Omega ) ,\\
&-\int_{\Omega} u \Delta \varphi + \int_{\Omega} g(u)\varphi \le \int_{\Omega}\varphi \, d\mu,\quad \forall \varphi \in C^2_0(\overline{\Omega}),\varphi \ge 0 .
\end{aligned}
\right.
\]

Note that a solution $u$ of \eqref{4main} satisfies $u\in L^1(\Omega)$ and $\Delta u \in \mathcal{M}(\Omega)$. So we can apply Kato's Inequality for solutions of $\eqref{4main}$.

We say that $\mu \in \mathcal{M}(\Omega)$ is a \emph{good measure} if \eqref{4main} admits a solution, and denote by $\mathcal{G}$ (depending on $g$) the set of good measures.

In this section, we are going to construct $u^*$ and $\mu ^*$. The first one is the largest subsolution. The second one, which we call \emph{reduced measure}, is the the largest good measure $\le \mu$.

From now, we suppose more that $N \ge 2$ and
\[
g(t) = 0,\quad \forall t \le 0.
\]

We say that a function $h:\mathbb{R} \to \mathbb{R}$ has \emph{subcritical growth} if there exist $C > 0$ and $p \in [1, \frac{N}{N-2})$ such that
\[
h(t) \le C(|t|^p + 1) ,\quad \forall t \in \mathbb{R}.
\]

Let $\{g_n\}$ be a sequence of continuous, non-decreasing and subcritical growth functions such that
\begin{equation}\label{g1}
0 \le g_{1}(t) \le g_{2}(t) \le \ldots \le g(t), \quad \forall t \in \mathbb{R},
\end{equation}
and
\begin{equation}\label{g2}
g_n(t) \to g(t) ,\quad \forall t \in \mathbb{R}.
\end{equation}

For example, we may take the following sequence $\{ g_n \}$:
\[
g_n(t) = \min \{g(t), n\},\quad \forall t \in \mathbb{R},\forall n.
\]

By Dini's Theorem, the conditions \eqref{g1} and \eqref{g2} imply that $g_n \to g$ uniformly on compact subsets of $\mathbb{R}$.

\begin{lmm}[Dini's Theorem]
Let $K$ be a compact topological space, and $\{f_n\}$ be a sequence of continuous real-valued functions on $K$ which converges pointwise to a continuous function $f$ and such that
\[
f_n(x) \le f_{n+1}(x), \quad \forall x\in K, \forall n.
\]
Then $\{f_n\}$ converges to $f$ uniformly.
\end{lmm}

By Theorem 2.6, for each $n$, there exists $u_n\in L^1(\Omega)$ such that
\begin{equation}\label{seqsol}
\left\{
\begin{aligned}
-\Delta u_n + g_n(u_n) & = \mu && \text{in } \Omega , \\
u_n & = 0 && \text{on } \partial \Omega.
\end{aligned}
\right.
\end{equation}

We come to the first result:

\begin{thm}\label{thm41}
Taking $n \to \infty$ in \eqref{seqsol}, we have $u_n \downarrow u^*$ in $\Omega$, where $u^*$ is the largest subsolution to \eqref{4main}. Moreover, we have
\begin{equation}\label{411}
\| g(u^*) \|_1 \le \| \mu \|_{\mathcal{M}},
\end{equation}
and
\begin{equation}\label{412}
\left| \int_{\Omega} u^* \Delta \varphi \right| \le 2 \|\mu\|_{\mathcal{M}}\|\varphi\|_{\infty},\quad \forall \varphi \in C^2_0(\overline{\Omega}).
\end{equation}
\end{thm}

First, we have to show that $\{u_n\}$ is a non-increasing sequence. This is drawn from the following lemma:

\begin{lmm}\label{cmplmm}
Let $g_1, g_2:\mathbb{R}\to\mathbb{R}$ be two continuous, non-decreasing functions such that $g_1 \le g_2$, and $u_1, u_2 \in L^1(\Omega)$ such that $g_1(u_1), g_2(u_2)\in L^1(\Omega)$. If
\[
-\int_{\Omega}(u_2 - u_1)\Delta \varphi + \int_{\Omega}[g_2(u_2)-g_1(u_1)]\varphi \le 0 ,\quad \forall \varphi \in C^2_0(\overline{\Omega}),\varphi\ge 0,
\]
then
\[
u_1 \ge u_2 \quad \text{a.e.\ in } \Omega.
\]
\end{lmm}

\begin{proof}
\mbox{}

Applying Kato's Inequality (Theorem \ref{kato1thm}) for 
\[
u = u_2 - u_1,
\]
and
\[
f = g_1(u_1) - g_2(u_2),
\]
we have
\begin{equation}\label{kause}
-\int_{\Omega}(u_2 - u_1)^+\Delta \varphi \le \int_{[u_2\ge u_1]}[g_1(u_1)-g_2(u_2)]\varphi ,\quad \forall \varphi \in C^2_0(\overline{\Omega}),\varphi\ge 0.
\end{equation}

On the set $[u_2 \ge u_1]$, we have
\[
g_2(u_2) \ge g_1(u_2) \ge g_1(u_1),
\]
which together with \eqref{kause} implies that
\[
-\int_{\Omega}(u_2 - u_1)^+\Delta \varphi \le 0 ,\quad \forall \varphi \in C_0^2(\overline{\Omega}),\varphi \ge 0.
\]

By Lemma \ref{L1e1lmm}, we obtain
\[
(u_2 - u_1)^+ \le 0 \quad \text{a.e.\ in } \Omega,
\]
which implies that
\[
(u_2 - u_1)^+ = 0 \quad \text{a.e.\ in }\Omega.
\]

So we have
\[
u_1 \ge u_2 \quad \text{a.e.\ in }\Omega.\qedhere
\]
\end{proof}

\begin{proof}[Proof of Theorem 1]
\mbox{}

By \eqref{g1} and the above lemma, $\{u_n\}$ is a non-increasing sequence.

Moreover, by Theorem \ref{apthm}, we get
\begin{equation}\label{413}
\| u_n \|_1 \le C \| \mu \|_{\mathcal{M}}, \quad \forall n,
\end{equation}
and
\begin{equation}\label{414}
\| g_n(u_n) \|_1 \le \| \mu \|_{\mathcal{M}}, \quad \forall n.
\end{equation}

Applying the monotone convergence theorem for the sequence $\{-u_n\}$, we get a function $u^* \in L^1(\Omega)$ such that
\[
u_n \to u^* \quad \text{a.e.\ in } \Omega,
\]
and
\begin{equation}\label{415}
u_n \to u^* \quad \text{in } L^1(\Omega).
\end{equation}

Let $x\in \Omega$ be such that $u_n(x)\to u^*(x)$.

By Dini's theorem, for each $\epsilon > 0$, we can choose $M_x$ (depending on $\epsilon$) large enough such that for every $n > M_x$,
\begin{equation}\label{141}
|g_n(t) - g(t)| < \epsilon / 2, \quad \forall t \in [u^*(x) - 1, u^*(x) + 1].
\end{equation}

Since $u_n(x)\to u^*(x)$ and $g$ is continuous, this $M_x$ can be chosen such that for every $n > M_x$,
\begin{equation}\label{142}
|u_n(x) - u(x)|  < 1,
\end{equation}
and
\begin{equation}\label{143}
|g(u_n(x)) - g(u(x))| < \epsilon / 2 .
\end{equation}

By \eqref{141}, \eqref{142} and \eqref{143}, we obtain that for every $n > M_x$,
\[
|g_n(u_n(x)) - g(u^*(x))| \le |g_n(u_n(x)) - g(u_n(x)| +|g(u_n(x)) - g(u^*(x)| \le \epsilon,
\]
which implies that
\begin{equation}\label{417}
g_n(u_n) \to g(u^*)\quad\text{a.e.\ in }\Omega.
\end{equation}

By Fatou's Lemma, we have $g(u^*) \in L^1(\Omega)$ and for all $\varphi \in C^2_0(\overline{\Omega})$,
\begin{align*}
\int_{\Omega} g(u^*)\varphi &\le \liminf_{n\to \infty} \int_{\Omega} g_n(u_n)\varphi \\
&= \liminf_{n\to \infty}\left( \int_{\Omega}u_n\Delta \varphi + \int_{\Omega}\varphi \,d\mu \right)\\
&= \int_{\Omega}u^* \Delta\varphi + \int_{\Omega}\varphi \,d\mu.
\end{align*}

In other words, $u^*$ is a subsolution to our problem.

To show that $u^*$ is the largest subsolution, let $v$ be any subsolution to our problem, we have to prove that $v \le u^*$ a.e.\ in $\Omega$.

Indeed, for each $n$, for every $\varphi \in C_0^2(\overline{\Omega})$ with $\varphi \ge 0$ in $\Omega$, we have
\begin{align*}
-\int_{\Omega}v \Delta \varphi + \int_{\Omega} g_n(v)\varphi &\le -\int_{\Omega}v \Delta \varphi + \int_{\Omega}g(v)\varphi \\
&\le \int_{\Omega} \varphi \, d\mu \\
& = -\int_{\Omega}u_n \Delta \varphi + \int_{\Omega} g_n(u_n)\varphi.
\end{align*}

By Lemma \ref{cmplmm}, for each $n$, we get
\[
v \le u_n \quad \text{a.e.\ in }\Omega.
\]

Let $n \to \infty$, we come to the conclusion that
\[
v \le u^* \quad \text{a.e.\ in } \Omega.
\]

Now, in \eqref{417}, using Fatou's Lemma, we get
\[
\int_{\Omega}|g(u^*)| \le \liminf_{n\to \infty} \int_{\Omega}|g_n(u_n)|,
\]
which together with \eqref{414} imply the estimate \eqref{411}.

Finally, for all $n$ and $\varphi \in C^2_0(\overline{\Omega})$, we have
\begin{align*}
\left|\int_{\Omega}u_n \Delta \varphi\right| &=  \left|\int_{\Omega} \varphi \,d\mu - \int_{\Omega}g_n(u_n)\varphi \right|\\
&\le (\| \mu \|_{\mathcal{M}} + \| g_n(u_n) \|_{\mathcal{M}})\| \varphi \|_{\infty}\\
&\le 2\| \mu\|_{\mathcal{M}} \| \varphi \|_{\infty},
\end{align*}

Taking $n \to \infty$, by \eqref{415}, the estimate \eqref{412} follows.
\end{proof}

\begin{rmk}
Because $u^*$ is the largest subsolution to our problem, it is not depend on the way we choose the sequence $\{g_n\}$.
\end{rmk}

\begin{crr}
If $\mu$ is a good measure, then $u^*$ coincides with the unique solution $u$.
\end{crr}
\begin{proof}
Suppose $\mu$ is a good measure, we have to prove $u = u^*$ a.e.\ in $\Omega$.

Because $u$ is a solution, it is also a subsolution. So, by Theorem \ref{thm41}, we have
\begin{equation}\label{leq1}
u \le u^*\quad \text{a.e.\ in }\Omega.
\end{equation}

Moreover, by applying Lemma \ref{cmplmm} for the estimate:
\begin{align*}
-\int_{\Omega}u^*\Delta\varphi + \int_{\Omega}g(u^*)\varphi &\le \int_{\Omega}\varphi\, d\mu \\
&= -\int_{\Omega}u\Delta\varphi + \int_{\Omega}g(u)\varphi, \quad\forall \varphi\in C^2_0(\overline{\Omega}),\varphi\ge 0,
\end{align*}
we obtain
\begin{equation}\label{leq2}
u^* \le u \quad \text{a.e.\ in } \Omega.\qedhere
\end{equation}

By \eqref{leq1} and \eqref{leq2}, we get the conclusion.
\end{proof}

Now, we ready to construct the \emph{reduced measure} $\mu^*$ and show that this is the largest good measure $\le \mu$.

\begin{crr}
There exists a unique measure $\mu^*\in \mathcal{M}(\Omega)$ such that
\begin{equation}\label{42crr}
-\int_{\Omega}u^*\Delta\varphi + \int_{\Omega}g(u^*)\varphi = \int_{\Omega} \varphi \,d\mu^*, \quad \forall \varphi \in C_0^2(\overline{\Omega}).
\end{equation}
\end{crr}

\begin{proof}
By \eqref{411} and \eqref{412}, it is clear that the functional
\[
T(\varphi) = -\int_{\Omega}u^*\Delta\varphi + \int_{\Omega}g(u^*)\varphi
\]
is a continuous linear functional on $C^2_0(\overline{\Omega})$ space (with norm $\| .\|_{\infty}$). Due to the density of $C^2_0(\overline{\Omega})$ space in $C_0(\overline{\Omega})$ space with norm $\| .\|_{\infty}$, there exists a unique extension of $T$ to $[C_0(\overline{\Omega})]^*$. Let $\mu^*$ be this extension. Clearly, $\mu^*$ satisfies \eqref{42crr}.
\end{proof}

\begin{thm}\label{thm42}
The reduced measure $\mu^*$ is the largest good measure $\le \mu$.
\end{thm}

Clearly, $\mu^*$ is a good measure. Moreover, since $u^*$ is a subsolution, we have $\mu^* \le \mu$. To show that this reduced measure is the largest, we use the following lemma:

\begin{lmm}\label{43lmm}
The reduced measure $\mu^*$ satisfies
\[
\mu^* \ge \mu_d - \mu_c^-.
\]
\end{lmm}

\begin{proof}
\mbox{}

Let $\{u_n\}$ be the sequence constructed in \eqref{seqsol}. For each $n$, we have
\[
(\Delta u_n)_d = g_n(u_n) - \mu_d \quad\text{and}\quad (\Delta u_n)_c = -\mu_c.
\]

By Corollary \ref{katocrr2}, we have
\begin{align*}
\Delta T_k(u_n) & \le \chi_{[u_n\le k]}(\Delta u_n)_d + (\Delta u_n)_c^+\\
&= \chi_{[u_n\le k]}( g_n(u_n)-\mu_d ) + (-\mu_c)^+\\
&\le g_n(T_k(u_n)) - \chi_{[u_n\le k]}\mu_d + \mu_c^-  \quad\text{in } \mathcal{D}'(\Omega),
\end{align*}
which implies
\begin{align*}
-\Delta T_k(u_n) + g_n(T_k(u_n)) &\ge \chi_{[u_n\le k]}\mu_d - \mu_c^- \\
&= \chi_{[u_n\le k]}\mu_d^+ - \chi_{[u_n\le k]}\mu_d^- - \mu_c^- \quad \text{in }\mathcal{D}'(\Omega).
\end{align*}

So, due to $u^* \le u_n \le u_1$ (from Theorem \ref{thm41}), we get
\begin{equation}\label{42thm1}
-\Delta T_k(u_n) + g_n(T_k(u_n)) \ge \chi_{[u_1 \le k]}\mu_d^+ - \chi_{[u^*\le k]}\mu_d^- -\mu_c^- \quad \text{in }\mathcal{D}'(\Omega).
\end{equation}

Moreover, by Dini's Theorem, we have
\[
g_n(T_k(u_n)) \to g(T_k(u^*)) \quad\text{a.e.\ in }\Omega,
\]
which imply (by the dominated convergence theorem) that as $n \to \infty$,
\[
g_n(T_k(u_n)) \to g(T_k(u^*)) \quad\text{in }L^1(\Omega).
\]

Let $n\to \infty$ in \eqref{42thm1}, we get
\[
-\Delta T_k(u^*) + g(T_k(u^*)) \ge \chi_{[u_1 \le k]}\mu_d^+ - \chi_{[u^*\le k]}\mu_d^- -\mu_c^- \quad \text{in }\mathcal{D}'(\Omega).
\]

Let $k \to \infty$. Since both sets $[u_1 = +\infty]$ and $[u^* = +\infty]$ have zero capacity, we conclude that 
\[
\mu^* = -\Delta u^* + g(u^*) \ge \mu_d^+ - \mu_d^- -\mu_c^- = \mu_d - \mu_c^-.\qedhere
\]
\end{proof}

The following consequence of the above lemma tells us that $\mu - \mu^*$ is concentrated on some set with zero capacity.

\begin{crr}\label{43crr}
There exists a Borel set $E \subset \Omega$ with $\capa(E) = 0$ such that
\[
(\mu - \mu^*)(\Omega \setminus E) = 0.
\]
\end{crr}

\begin{proof}
By Lemma \ref{43lmm}, we have
\[
0 \le \mu -\mu^* \le \mu - \mu_d + \mu_c^- = \mu_c^+,
\]
which gives us the conclusion.
\end{proof}

\begin{proof}[Proof of Theorem \ref{thm42}]
Let $\nu$ be a good measure $\le \mu$. We need to show that $\nu \le \mu^*$.

Denote by $v$ the solution of \eqref{4weak} corresponding to $\nu$ instead of $\mu$.

By Corollary \ref{43crr}, $\mu - \mu^*$ is the ``concentrated'' part of itself. So,
\[
(\mu - \mu^*)_d = 0,
\]
or in other words,
\[
(\mu^*)_d = \mu_d.
\]

Due to $\nu \le \mu$, we have
\begin{equation}\label{thm411}
\nu_d \le \mu_d = (\mu^*)_d.
\end{equation}

Moreover, by Lemma \ref{cmplmm}, we have
\[
u^* - v \ge 0 \quad\text{a.e.\ in }\Omega.
\]

By Corollary \ref{katocrr1}, we get
\[
(-\Delta (u^* -v))_c \ge 0,
\]
which implies
\[
\nu_c = (-\Delta v + g(v))_c = (-\Delta v)_c \le (-\Delta u^*)_c = (\mu^*)_c \quad\text{in }\mathcal{D}'(\Omega),
\]
or in other words,
\begin{equation}\label{thm412}
\nu_c \le (\mu^*)_c.
\end{equation}

By \eqref{thm411} and \eqref{thm412}, we get the conclusion.
\end{proof}

\section{\texorpdfstring{Some properties of $\mathcal{G}$}{Some properties of G}}
\mbox{}

In this section, we will give some results about good measures. These results are almost all obtained by using the new concept \emph{reduced measures}, which is constructed in the above section.

We start with the following theorem:

\begin{thm}\label{gcmpthm}
Let $\mu_1, \mu_2 \in \mathcal{M}(\Omega)$. If  $\mu_1\in \mathcal{G}$ and $\mu_2 \le \mu_1$, then $\mu_2 \in \mathcal{G} $.
\end{thm}

To prove this theorem, we need the following lemma:

\begin{lmm}\label{23lmm}
If $\mu$ is a good measure with solution $u$, and $u_n$ is given by \eqref{seqsol}, then
\[
\quad g_n(u_n) \to g(u) \quad\text{in }L^1(\Omega).
\]
\end{lmm}

\begin{proof}
\mbox{}

For each $n$, we have
\[
-\Delta(u_n -u) = g(u) - g_n(u_n) \quad\text{in }[C^2_0(\overline{\Omega})],
\]

Apply Lemma \ref{L1e3lmm} for the above equation, we have
\[
\int_{\Omega} (g(u)-g_n(u_n)).\sgn(u_n-u) \ge 0,
\]
which implies
\[
\int_{\Omega} (g_n(u_n)-g_n(u)).\sgn(u_n-u) \le \int_{\Omega}(g(u)-g_n(u)).\sgn(u_n -u).
\]

Hence
\[
\| g_n(u_n) - g_n(u) \|_1 \le \| g(u)-g_n(u)\|_1,
\]
which implies
\[
\| g_n(u_n) - g(u) \|_1 \le 2\| g(u)-g_n(u)\|_1.
\]

In other words, $g_n(u_n) \to g(u)$ in $L^1(\Omega)$.
\end{proof}

\begin{proof}[Proof of Theorem \ref{gcmpthm}]
Suppose that $\mu_1$ is a good measure with solution $u_1$.

For each $n$, let $u_{i,n}$ be the solution of
\[
\left\{
\begin{aligned}
-\Delta u_{i,n} + g_n(u_{i,n}) & = \mu_i && \text{in } \Omega , \\
u_{i,n} & = 0 && \text{on } \partial \Omega,
\end{aligned}
\right.
\]
for $i = 1,2$.

Since $\mu_2 \leq \mu_1$, using Lemma \ref{cmplmm}, we obtain
\[
u_{2,n} \le u_{1,n} \quad\text{a.e.\ in }\Omega,
\]
which implies
\[
g_n(u_{2,n})\le g_n(u_{1,n})\quad\text{a.e.\ in }\Omega.
\]

Moveover, by Lemma \ref{23lmm}, we have
\[
g_n(u_{1,n}) \to g(u_1)\quad\text{in } L^1(\Omega).
\]

Therefore,
\[
g_n(u_{2,n}) \to g(u_2^*) \quad\text{in }L^1(\Omega),
\]
where $u_2^*$ is the largest subsolution of the problem with measure data $\mu_2$.

Adapting the proof of Theorem \ref{thm41}, we conclude that $\mu_2$ is a good measure with the solution $u_2^*$.
\end{proof}

Here is some consequences of the above theorem.

\begin{crr}
Let $\mu \in \mathcal{M}(\Omega)$. If $\mu \leq 0$, then $\mu \in \mathcal{G}$.
\end{crr}

\begin{proof}
By Theorem \ref{L1thm}, the trivial measure $0$ is a good measure. So the conclusion follows Theorem \ref{gcmpthm}.
\end{proof}

\begin{crr}\label{crr45}
If $\mu_1, \mu_2 \in \mathcal{G}$, then $\sup \{ \mu_1, \mu_2 \}\in \mathcal{G}$.
\end{crr}

\begin{proof}
Let $\nu = \sup \{ \mu_1, \mu_2 \}$. By Theorem \ref{thm42}, we have $\mu_1 \le \nu^*$ and $\mu_2 \le \nu^*$. So, $\nu \le \nu^*$, which means $\nu =\nu^* \in \mathcal{G}$.
\end{proof}

\begin{crr}\label{crr46}
Let $\mu \in \mathcal{M}(\Omega)$. We have
\[
\mu \in \mathcal{G} \quad \Leftrightarrow \quad \mu^+ \in \mathcal{G} \quad \Leftrightarrow \quad \mu_c \in \mathcal{G}.
\]
\end{crr}

\begin{proof}
The implications in this corollary can be obtained by the following facts:
\[
\mu^+ = \sup\{\mu, 0\},
\]
and
\[
\mu^+ = \sup\{\mu_d, \mu_c\},
\]
with a note that $0$ and $\mu_d$ are good measures.
\end{proof}

\begin{crr}\label{crr47}
We have
\[
\mathcal{G} + \mathcal{M}_d(\Omega) \subset \mathcal{G}.
\]
In particular,
\[
\mathcal{G} + L^1(\Omega) \subset \mathcal{G}.
\]
\end{crr}

\begin{proof}
Let $\mu \in \mathcal{G}$ and $\nu \in \mathcal{M}_d(\Omega)$. By Corollary \ref{crr46}, we have $\mu_c \in \mathcal{G}$. Since
\[
(\mu+\nu)_c = \mu_c + \nu_c = \mu_c,
\]
we obtain $(\mu+\nu)_c \in \mathcal{G}$. The conclusion follows. 
\end{proof}

\begin{crr}
The set $\mathcal{G}$ is convex.
\end{crr}

\begin{proof}
Let $\mu_1, \mu_2 \in \mathcal{G}$. For any $t \in [0,1]$, we have 
\[
t\mu_1 + (1-t)\mu_2 \le \sup \{\mu_1, \mu_2\}.
\]
By Theorem \ref{gcmpthm} and Corrollary \ref{crr45}, we get $t\mu_1 + (1-t)\mu_2 \in \mathcal{G}$. The conclusion follows.
\end{proof}

\begin{thm}
The set $\mathcal{G}$ is closed with respect to strong convergence in $\mathcal{M}(\Omega)$.
\end{thm}

\begin{proof}
Let $\{\mu_n\}$ be a sequence of good measures such that $\mu_n \to \mu$ strongly in $\mathcal{M}(\Omega)$, and for each $n$, let $u_n$ be the solution to the problem:
\begin{equation}\label{44thm1}
\left\{
\begin{aligned}
-\Delta u_{n} + g(u_{n}) & = \mu_n && \text{in } \Omega , \\
u_{n} & = 0 && \text{on } \partial \Omega.
\end{aligned}
\right.
\end{equation}

For each $m,n$, by Theorem \ref{apthm}, we have
\[
\| g(u_m)-g(u_n)\|_1 \leq \| \mu_m - \mu_n \|_{\mathcal{M}},
\]
and
\[
\| u_m-u_n\|_1 \leq C\| \mu_m - \mu_n \|_{\mathcal{M}}.
\]

So both $\{u_n\}$ and $\{g(u_m)\}$ are Cauchy sequences in $L^1(\Omega)$. Thus there exists $u \in L^1(\Omega)$ such that
\[
u_n \to u \text{ in }L^1(\Omega) \quad\text{and} \quad g(u_n)\to g(u) \text{ in } L^1(\Omega).
\]

Let $n \to \infty$ in \eqref{44thm1}, we get the conclusion that $\mu$ is a good measure with solution $u$.
\end{proof}

\begin{thm}\label{thm45}
Let $\mu \in \mathcal{M}(\Omega)$. We have $\mu \in \mathcal{G}$ if and only if there exist $f_0 \in L^1(\Omega)$ and $v_0 \in L^1(\Omega)$ such that $g(v_0) \in L^1(\Omega)$ and
\[
\mu = f_0 - \Delta v_0 \quad \text{in } \mathcal{D}'(\Omega),
\]
which is
\begin{equation}\label{thm441}
\int_{\Omega} \varphi \,d\mu = \int_{\Omega}f_0\varphi - \int_{\Omega}v_0\Delta \varphi, \quad \forall \varphi \in C_c^{\infty}(\Omega).
\end{equation}
\end{thm}

\begin{proof}
Put $f_0 = g(v_0)$, the ``only if'' implication is clear. So we just have to prove the ``if'' side.

First, suppose that $v_0$ has compact support $K \subset \Omega$ and let $\varphi \in C^2_0(\overline{\Omega})$. By ``convolution'', we can construct a sequence $\{\varphi_n\}\subset C_c^{\infty}(\Omega)$ such that
\begin{equation}\label{thm4524}
\varphi_n \to \varphi \text{ in }L^{\infty}(\Omega) \quad\text{and}\quad \Delta\varphi_n \rightrightarrows \Delta\varphi \text{ uniformly in }K.
\end{equation}

The last one implies that
\[
\int_K v_0 \Delta \varphi_n \to \int_K v_0 \Delta \varphi.
\]

Since $v_0 = 0$ a.e.\ in $\Omega\setminus K$, we have
\[
\int_{\Omega} v_0 \Delta \varphi_n \to \int_{\Omega} v_0 \Delta \varphi.
\]

By \eqref{thm441}, we have
\[
\int_{\Omega} \varphi_n \,d\mu = \int_{\Omega}f_0\varphi_n - \int_{\Omega}v_0\Delta \varphi_n,\quad\forall n.
\]

Let $n \to \infty$, we obtain
\[
\int_{\Omega} \varphi \,d\mu = \int_{\Omega}f_0\varphi - \int_{\Omega}v_0\Delta \varphi .
\]

Therefore,
\[
\int_{\Omega} \varphi \,d\mu + \int_{\Omega}(g(v_0)-f_0)\varphi = -\int_{\Omega}v_0\Delta \varphi + \int_{\Omega}g(v_0)\varphi.
\]

Because $\varphi$ is chosen arbitrarily in $C_0^2(\overline{\Omega})$, we conclude that $\mu + g(v_0) - f_0$ is a good measure. Thus, by Corollary \ref{crr47}, $\mu$ is a good measure.

Now, for general $v_0$, since
\[
\Delta v_0 = \mu - f_0 \quad\text{in }\mathcal{D}'(\Omega),
\]
we have $v_0 \in W^{1,1}_{loc}(\Omega)$ (see \cite[Lemma 5.3]{P}).

For each $\psi \in C_c^{\infty}(\Omega$, we have
\[
\int_{\Omega}\varphi \psi\,d\mu = \int_{\Omega}f_0 \varphi \psi - \int_{\Omega}v_0 \Delta(\varphi \psi), \quad\forall\varphi \in C_c^{\infty}(\Omega).
\]

Since $\Delta (\varphi \psi)  = \varphi \Delta \psi + \psi \Delta \varphi + 2\nabla \varphi.\nabla \psi$ and
\begin{align*}
\int_{\Omega}v_0 \nabla \varphi. \nabla \psi &= \int_{\Omega} \nabla \varphi. (v_0\nabla \psi)\\
&= \int_{\Omega} \varphi \dive(v_0\nabla \psi)\\
&= \int_{\Omega} \varphi (\nabla v_0.\nabla \psi + v_0\Delta \psi),
\end{align*}
we have
\[
\int_{\Omega}\varphi \psi\,d\mu = \int_{\Omega}(f_0\psi - v_0\Delta \psi - 2\nabla v_0.\nabla\psi)\varphi - \int_{\Omega}\psi v_0 \Delta\varphi, \quad\forall\varphi \in C_c^{\infty}(\Omega),
\]
which means that
\begin{equation}\label{481}
\psi\mu = (f_0\psi - v_0\Delta\psi - 2\nabla v_0.\nabla\psi) - \Delta(\psi v_0) \quad\text{in } \mathcal{D}'(\Omega),
\end{equation}
where $\psi\mu$ denotes the following functional in $[C_0(\overline{\Omega}]^*$:
\[
\varphi \mapsto \int_{\Omega}\varphi\psi\,d\mu.
\]

Let $\{\psi_n\}\subset C_c^{\infty}(\Omega)$ be such that for each $n$, $0\leq \psi_n \leq 1$ in $\Omega$ and $\psi_n(x) = 1$ if $d(x,\partial \Omega)> \frac{1}{n}$. By \eqref{481}, for each $n$, we have
\[
\psi_n\mu = (f_0\psi - v_0\Delta\psi_n - 2\nabla v_0.\nabla\psi_n) - \Delta(\psi_n v_0) \quad\text{in } \mathcal{D}'(\Omega).
\]

Since $f_0\psi - v_0\Delta\psi_n - 2\nabla v_0.\nabla\psi_n \in L^1(\Omega)$ and $\psi_n v_0$ has compact support in $\Omega$, $\psi_n\mu$ is a good measure (with a note that $g(\psi_n v_0) \in L^1(\Omega)$ since $0\leq g(\psi_n v_0)\leq g(v_0)$).

It is not difficult to show that $\psi_n\mu$ converges to $\mu$ strongly in $\mathcal{M}(\Omega)$. Since $\mathcal{G}$ is closed with respect to strong convergence in $\mathcal{M}(\Omega)$, the conclusion that $\mu$ is a good measure follows. 
\end{proof}

\begin{crr}
Assume
\[
g(2t) \le C(g(t) +1), \quad \forall t\ge 0.
\]
Then the set $\mathcal{G}$ is a convex cone.
\end{crr}

\begin{proof}
Let $\mu$ be a good measure with solution $u$. Clearly, it suffices to show that $2\mu$ is a good measure. Since
\[
\mu = g(u) - \Delta u \quad\text{in }\mathcal{D}'(\Omega),
\]
we have
\[
2\mu = 2g(u) - \Delta(2u) \quad\text{in }\mathcal{D}'(\Omega).
\]
By assumption, we have $g(2u)$ in $L^1(\Omega)$. Now, in Theorem \ref{thm45}, replace $\mu$, $f_0$, and $v_0$ by $2\mu$, $2g(u)$, and $2u$ respectively, we get the conclusion.
\end{proof}

Now, we come to an interesting result. It tells that every nonnegative measure which is good for every $g$ must be diffuse. For the proof, see the main article of this report \cite[Section 6]{BMP}.

\begin{thm}
Let $\mu \in \mathcal{M}(\Omega)$. We have
\[
\mu \in \mathcal{G}(g) \text{ for every } g \quad \Leftrightarrow \quad \mu^+ \in \mathcal{M}_d(\Omega).
\]
\end{thm}

\section{\texorpdfstring{Some properties of $\mu^*$}{Some properties of mu*}}
\mbox{}

Here we state some properties of reduced measures.

\begin{prn}
We have
\[
0 \le \mu - \mu ^* \le \mu^+.
\]
In particular,
\begin{equation}\label{016}
|\mu ^*| \le |\mu|,
\end{equation}
and
\[
\mu \ge 0 \quad \Rightarrow \quad \mu ^* \ge 0.
\]
\end{prn}

\begin{proof}
\mbox{}

Since every nonpositive measure is a good measure, we have
\[
-\mu^- \in\mathcal{G},
\]
which implies that
\[
-\mu^- \leq \mu^*.
\]

In other words,
\[
\mu - \mu^* \leq \mu^+.
\]

Now, since
\[
-|\mu| \leq -\mu^- \leq \mu^* \leq \mu \leq |\mu|,
\]
we have
\[
|\mu^*|\leq |\mu|.
\]

Finally, suppose that $\mu \geq 0$. Since $\mu = \mu^+$, we obtain $\mu^* \ge 0$.
\end{proof}

\begin{prn}
Let $\mu \in \mathcal{M}(\Omega)$. We have
\begin{equation}\label{prn50}
\| \mu - \mu^*\|_{\mathcal{M}} = \min_{\nu \in \mathcal{G}} \| \mu - \nu \|_{\mathcal{M}}.
\end{equation}
Moreover, $\mu^*$ is the unique good measure which achieves the minimum.
\end{prn}

\begin{proof}
Let $\nu \in\mathcal{G}$.

Since $\inf\{\mu,\nu\} \leq \nu$, we have 
\[
\inf\{\mu,\nu\} \in \mathcal{G}.
\]

So, by $\inf\{\mu,\nu\} \leq \mu$, we have
\begin{equation}\label{prn501}
\inf\{\mu,\nu\} \leq \mu^*.
\end{equation}

Moreover,
\begin{equation}\label{prn502}
|\mu - \nu| = (\mu - \nu)^+ + (\mu - \nu)^- \geq (\mu -\nu)^+ = \mu - \inf\{\mu,\nu\}.
\end{equation}

Combining \eqref{prn501} and \eqref{prn502}, we get
\[
|\mu - \nu| \ge \mu - \inf\{\mu,\nu\} \ge \mu -\mu^*  \geq 0,
\]
we obtain
\[
\| \mu - \mu^*\|_{\mathcal{M}} \le \| \mu - \nu \|_{\mathcal{M}},
\]
which implies \eqref{prn50}.

Now, for the uniqueness, let $\nu$ is a good measure which also achieves the minimum. Then, we must have the inequalities in \eqref{prn501} and \eqref{prn502} attained the equal signs. So,
\[
\nu = \inf\{\mu,\nu\}\quad\text{and}\quad \inf\{\mu,\nu\} = \mu^*,
\]
which imply that $\nu = \mu^*$.
\end{proof}

\begin{prn}\label{prn43}
Let $\mu, \nu \in \mathcal{M}(\Omega)$. If $\mu \le \nu$, then $\mu^* \le \nu^*$.
\end{prn}

\begin{proof}
This proposition is a consequence of Theorem \ref{gcmpthm} since $\mu^*$ is a good measure and
\[
\mu^* \leq \mu \leq \nu.\qedhere
\]
\end{proof}

\begin{thm}\label{thm47}
Let $\mu_1,\mu_2 \in \mathcal{M}(\Omega)$. If $\mu_1,\mu_2$ are mutually singular, then
\[
(\mu_1 + \mu_2)^* = (\mu_1)^* + (\mu_2)^*.
\]
\end{thm}

\begin{proof}
\mbox{}

By \eqref{016}, we also have $(\mu_1)^*$ and $(\mu_2)^*$ mutually singular. In particular, we have
\[
(\mu_1)^* + (\mu_2)^* \leq [(\mu_1)^* + (\mu_2)^*]^+ = \sup \{(\mu_1)^* , (\mu_2)^*\},
\]
which implies that $(\mu_1)^* + (\mu_2)^*$ is a good measure.

Since
\[
(\mu_1)^* + (\mu_2)^* \leq \mu_1 + \mu_2,
\]
we conclude that
\[
(\mu_1)^* + (\mu_2)^* \leq (\mu_1 + \mu_2)^*.
\]

Now, we prove the reverse inequality. First, let $\nu = (\mu_1 + \mu_2)^*$. By Lebesgue's Decomposition Theorem, we may decompose $\nu$ in terms of two measures:
\[
\nu = \nu_0 + \nu',
\]
where $\nu_0$ is singular with respect to $|\mu_1| + |\mu_2|$ and $\nu'$ is absolutely continuous with respect to $|\mu_1| + |\mu_2|$.

It is not difficult to show that $\nu' \leq \nu^+$. So we have $\nu'$ is a good measure, which implies that $\nu'^+$ is a good measure.

Moreover, since $\mu_1$ and $\mu_2$ are mutually singular, we may decompose $\nu'$ in terms of two measures:
\[
\nu' = \nu_1 + \nu_2,
\]
where $\nu_1$ is absolutely continuous with respect to $|\mu_1|$ and $\nu_2$ is absolutely continuous with respect to $|\mu_2|$.

Similarly, we have $\nu_1 \leq \nu'^+$ and $\nu_2 \leq \nu'^+$. Since $\nu'^+$ is a good measure, we have $\nu_1$ and $\nu_2$ are good measures.

By $\nu \leq \mu_1 + \mu_2$, we obtain
\[
\nu_0 \leq 0,\quad \nu_1 \leq \mu_1, \quad\text{and}\quad \nu_2 \leq \mu_2.
\]

Thus we have $\nu_1 \leq (\mu_1)^*$ and $\nu_2 \leq (\mu_2)^*$. The conclusion follows
\[
\nu = \nu_0 + \nu_1 + \nu_2 \leq (\mu_1)^* + (\mu_2)^*.\qedhere
\]
\end{proof}

\begin{crr}\label{crr410}
Let $\mu \in \mathcal{M}(\Omega)$. We have
\[
(\mu^*)_d = (\mu_d)^* = \mu_d \quad \text{and}\quad (\mu^*)_c = (\mu_c)^*.
\]
\end{crr}

\begin{proof}
By using Theorem \ref{thm47}, we have
\[
\mu^* = (\mu_d + \mu_c)^* = (\mu_d)^* + (\mu_c)^*,
\]
which gives us the conclusion due to
\[
\mu^* = (\mu^*)_d + (\mu^*)_c,
\]
and the uniqueness of the decomposition a measure into ``diffuse'' part and ``concentrated'' part.
\end{proof}

For convenience, we will denote $(\mu^*)_d$ by $\mu^*_d$ and $(\mu^*)_c$ by $\mu^*_c$.

\begin{crr}
Let $\mu \in \mathcal{M}(\Omega)$. We have
\[
(\mu^*)^+ = (\mu^+)^* \quad \text{and} \quad (\mu^*)^- = \mu^-.
\]
\end{crr}

\begin{proof}
Similar to Corollary \ref{crr410}, the conclusion comes from
\[
\mu^* = (\mu^+ - \mu^-)^* = (\mu^+)^* + (-\mu^-)^* = (\mu^+)^* - \mu^-,
\]
with a note that $-\mu^-$ is a good measure.
\end{proof}

\begin{crr}
Let $\mu \in \mathcal{M}(\Omega)$ and $\nu \in \mathcal{M}_d(\Omega)$. We have
\[
(\mu + \nu)^* = \mu^* + \nu.
\]
\end{crr}

\begin{proof}
The conclusion follows
\begin{align*}
(\mu + \nu)^* &= (\mu_c + \mu_d + \nu)^*\\
&= \mu_c^* + (\mu_d + \nu)^*\\
&= \mu_c^* + \mu_d + \nu \\
&= \mu_c^* + \mu_d^* + \nu \\
&= \mu^* + \nu.\qedhere
\end{align*}
\end{proof}

\begin{crr}\label{crr11}
Let $\mu \in \mathcal{M}(\Omega)$. For every Borel set $E \subset \Omega$, we have
\[
(\mu\lfloor_E)^* = \mu^*\lfloor_E.
\]
\end{crr}

\begin{proof}
The conclusions follows
\[
\mu^* = (\mu\lfloor_E + \mu\lfloor_{\Omega\setminus E})^* = (\mu\lfloor_E)^* + (\mu\lfloor_{\Omega\setminus E})^*. \qedhere
\]
\end{proof}

\begin{thm}
Let $\mu, \nu\in\mathcal{M}(\Omega)$. We have
\begin{equation}\label{infid}
[\inf\{\mu,\nu \}]^* = \inf\{\mu^*,\nu^*\},
\end{equation}
and
\begin{equation}\label{supid}
[\sup\{\mu,\nu\}]^* = \sup\{\mu^*,\nu^*\}.
\end{equation}
\end{thm}

\begin{proof}
\mbox{}

Since $\inf \{\mu^*,\nu^*\} \le \mu^*$, we have $\inf \{\mu^*,\nu^*\}$ is a good measure. Moreover, by
\[
\inf \{\mu^*,\nu^*\} \leq \mu^* \leq \mu,
\]
and
\[
\inf \{\mu^*,\nu^*\} \leq \nu^* \leq \nu,
\]
we get
\[
\inf \{\mu^*,\nu^*\} \leq \inf \{\mu,\nu\}.
\]

Therefore,
\[
\inf \{\mu^*,\nu^*\} \le [\inf \{\mu,\nu\}]^*.
\]

Reversely, by Proposition \ref{prn43}, we have
\[
[\inf\{\mu,\nu\}]^* \leq \mu^*,
\]
and
\[
[\inf\{\mu,\nu\}]^* \leq \nu^*,
\]
which imply that
\[
[\inf \{\mu,\nu\}]^* \le \inf \{\mu^*,\nu^*\} .
\]

So we have \eqref{infid}.

For the identity \eqref{supid}, applying Hahn Decomposition Theorem to $\mu - \nu$, we may write $\Omega$ in terms of two disjoint Borel sets $E_1,E_2\subset\Omega$, $\Omega = E_1 \cup E_2$, such that
\[
\mu \geq \nu \text{ in }E_1\quad\text{and}\quad \nu \geq \mu \text{ in }E_2.
\]

Then, we have
\begin{equation}\label{sup1}
\sup\{\mu,\nu\} = \mu + (\nu-\mu)^+ = \mu + (\nu-\mu)\lfloor_{E_2} = \mu\lfloor_{E_1} + \nu\lfloor_{E_2},
\end{equation}

By Corollary \ref{crr11}, we have
\[
\mu^*\lfloor_{E_1} = (\mu\lfloor_{E_1})^* \geq (\nu\lfloor_{E_1})^* = \nu^*\lfloor_{E_1}.
\]

Similarly, we also have
\[
\nu^*\lfloor_{E_2} \geq \mu^*\lfloor_{E_2} .
\]

With the same argument as in \eqref{sup1}, we obtain
\begin{equation}\label{sup2}
\sup\{\mu^*,\nu^*\} = \mu^*\lfloor_{E_1} + \nu^*\lfloor_{E_2}.
\end{equation}

On the other hand, by Theorem \ref{thm47}, we have
\begin{align*}
(\mu\lfloor_{E_1} + \nu\lfloor_{E_2})^*& = (\mu\lfloor_{E_1})^* + (\nu\lfloor_{E_2})^* \\
&= \mu^*\lfloor_{E_1} + \nu^*\lfloor_{E_2},
\end{align*}
which together with \eqref{sup1} and \eqref{sup2} imply that
\[
\sup\{\mu^*,\nu^*\} = [\sup\{\mu,\nu\}]^*.\qedhere
\]
\end{proof}

\begin{thm}
Let $\mu, \nu \in\mathcal{M}(\Omega)$. We have
\[
(\mu^* - \nu^*)^+ \le (\mu -\nu)^+.
\]
In particular,
\[
|\mu^* - \nu^*| \le |\mu - \nu|.
\]
\end{thm}

\begin{proof}
\mbox{}

First, we suppose that $\nu \leq \mu$.

For each $n$, let $v_n$ be the solution of \eqref{seqsol} corresponding to the measure $\nu$.

Since $\nu \leq \mu$, by Lemma \ref{cmplmm}, we have
\[
v_n \leq u_n \quad\text{a.e.\ in }\Omega.
\]

Thus,
\begin{equation}\label{ddfe}
g_n(v_n) \leq g_n(u_n)\quad\text{a.e.\ in }\Omega.
\end{equation}

Moreover, we have
\[
\int_{\Omega}(g_n(u_n)-g(u^*))\varphi = \int_{\Omega} (u_n-u^*) \Delta\varphi + \int_{\Omega}\varphi \,d(\mu - \mu^*),\quad\forall \varphi \in C^2_0(\overline{\Omega}),
\]
and
\[
\int_{\Omega}(g_n(v_n)-g(u^*))\varphi = \int_{\Omega} (v_n-v^*) \Delta\varphi + \int_{\Omega}\varphi \,d(\nu - \nu^*),\quad\forall \varphi \in C^2_0(\overline{\Omega}),
\]
which imply that
\[
\lim\limits_{n\to\infty} \int_{\Omega}(g_n(u_n)-g(u^*))\varphi = \int_{\Omega}\varphi \,d(\mu - \mu^*),\quad\forall \varphi \in C^2_0(\overline{\Omega}),
\]
and
\[
\lim\limits_{n\to\infty} \int_{\Omega}(g_n(v_n)-g(v^*))\varphi = \int_{\Omega}\varphi \,d(\nu - \nu^*),\quad\forall \varphi \in C^2_0(\overline{\Omega}),
\]

By \eqref{ddfe}, for each $n$, we have
\[
 \int_{\Omega}(g_n(v_n)-g(v^*))\varphi \leq  \int_{\Omega}(g_n(u_n)-g(u^*))\varphi, \quad \forall \varphi \in C^2_0(\overline{\Omega}),\varphi\geq 0.
\]

Taking $n \to \infty$, we obtain
\[
\int_{\Omega}\varphi \,d(\nu - \nu^*) \leq \int_{\Omega}\varphi \,d(\mu - \mu^*), \quad\forall \varphi \in C^2_0(\overline{\Omega}),\varphi\geq 0.
\]

So we have
\[
\nu - \nu^* \leq \mu - \mu^*,
\]
or in other words,
\[
\mu^* - \nu^* \leq \mu - \nu.
\]

We get the implication that
\[
\nu\leq \mu \quad\Rightarrow\quad \mu^* - \nu^* \leq \mu - \nu.
\]

Now, we remove the assumption $\nu \leq \mu$. By $\nu \leq \sup\{\mu,\nu\}$, applying the above result, we have
\[
[\sup\{\mu,\nu\}]^* - \nu^* \leq \sup\{\mu,\nu\} - \nu,
\]

Since
\[
[\sup\{\mu,\nu\}]^*=\sup\{\mu^*,\nu^*\},
\]
we obtain
\[
\sup\{\mu^*,\nu^*\} - \nu^* \leq \sup\{\mu,\nu\} - \nu.
\]

In other words,
\begin{equation}\label{fi5}
(\mu^* - \nu^*)^+ \leq (\mu - \nu)^+.
\end{equation}

Exchange the role of $\mu$ and $\nu$ in the above inequality, we get
\[
(\nu^* - \mu^*)^+ \leq (\nu - \mu)^+.
\]

In other words,
\begin{equation}\label{fi6}
(\mu^* - \nu^*)^- \leq (\mu - \nu)^-.
\end{equation}

By \eqref{fi5} and \eqref{fi6}, we obtain
\[
|\mu^* - \nu^*| \le |\mu - \nu|.\qedhere
\]
\end{proof}

\newpage
\begin{thebibliography}{99}
\bibitem{BMP} H. Brezis, M. Marcus and A.C. Ponce, \textit{Nonlinear elliptic equations with measures revisited}, Annals of Math. Studies 163, Princeton University Press, NJ, 2007, pp. 55--110.
\bibitem{P} A.C. Ponce, \textit{Selected problems on elliptic equations involving measures}, manuscript submitted to the concours annuel 2012 of the Acad\'emie royale de Belgique, \href{http://arxiv.org/abs/1204.0668v1}{arXiv:1204.0668v1} \textbf{[math.AP]}.
\bibitem{BS} H. Brezis and W.A. Strauss, \textit{Semilinear second-order elliptic equation in $L^1$}, J. Math. Soc. Japan \textbf{25} (1973), 565--590.
\bibitem{S} G. Stampacchia, \textit{\'Equations elliptiques du second ordre \`a coefficients discontinus}, Les Presses de l'Universit\'e de Montr\'eal, Montr\'eal, 1966.
\bibitem{BGO} L. Boccardo, T. Gallou\"et and L. Orsina, \textit{Existence and uniqueness of entropy solutions for nonlinear elliptic equations with measure data}, Ann. Inst. H. Poincar\'e Anal. Non Lin\'eaire \textbf{13} (1996), 539--551.
\bibitem{BP} H. Brezis and A.C. Ponce, \textit{Kato's inequality when $\Delta u$ is a measure}, C. R. Acad. Sci. Paris, Ser. I \textbf{338} (2004), 599--604.
\bibitem{BP2} H. Brezis and A.C. Ponce, \textit{Remarks on the strong maximum principle}, Diff. Int. Eq. \textbf{16} (2003), 1--12.
\end{thebibliography}

\end{document}
